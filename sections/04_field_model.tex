%==============================================================================
% Section 4: Minimal Field Model
%==============================================================================

\chap{Minimal Field Model}

This chapter states a minimal effective field model for a dimensionless scalar
$S(x)$ and fixes the notation used in the remainder of the document.
The model is used only as a weak-field, galaxy-scale effective description.

\subsection{Field variable and conventions}

We take $S(x)$ to be dimensionless. Indices are raised and lowered with the
background metric; $\nabla_\mu$ denotes the covariant derivative and
$\Box \equiv \nabla_\mu \nabla^\mu$.
In quasi-static applications we use the working relation
\begin{equation}
g_{\rm tot} \equiv c^2 \nabla S .
\end{equation}

\subsection{Minimal effective Lagrangian}

A compact effective Lagrangian density is
\begin{equation}
\mathcal{L}
=
\frac{\sigma}{2}\,\nabla_\mu S\,\nabla^\mu S
- V(S)
+ J(x)\,S,
\label{eq:L_min}
\end{equation}
where $\sigma$ is a constant coefficient, $V(S)$ is an effective potential,
and $J(x)$ denotes an external source term used to encode phenomenological
anchoring by baryonic structure.
No microscopic interpretation of $\sigma$, $V$, or $J$ is assumed in these notes.

\subsection{Equation of motion}

Varying the action associated with Eq.~\eqref{eq:L_min} yields
\begin{equation}
\sigma\,\Box S = J(x) - V'(S).
\label{eq:EOM_linear}
\end{equation}

For the galaxy-scale applications considered here, we primarily use the
quasi-static limit in which time derivatives are negligible and sources vary
slowly.
In that limit, Eq.~\eqref{eq:EOM_linear} reduces to a Poisson-like form,
\begin{equation}
\sigma\,\nabla^2 S \simeq J(x) - V'(S).
\label{eq:EOM_static}
\end{equation}

\subsection{Nonlinear ``vessel'' form}

To accommodate RAR-compatible phenomenology, it is convenient to generalize the
kinetic sector to a nonlinear response function $\mu(\mathcal{X})$ with
\begin{equation}
\mathcal{X} \equiv \frac{\nabla_\mu S\,\nabla^\mu S}{a_0^2},
\end{equation}
leading to the effective field equation
\begin{equation}
\nabla_\mu\!\Big(\mu(\mathcal{X})\,\nabla^\mu S\Big) - V'(S)
= -\kappa\,\rho_{\rm b}c^2 ,
\label{eq:EOM_nonlinear}
\end{equation}
where $\rho_{\rm b}$ is the baryonic rest-mass density used as the observational
source input and $\kappa$ is an effective coupling constant.
The function $\mu(\mathcal{X})$ is chosen so that the quasi-static limit matches
the empirical interpolation used for the observational benchmarks in Chapter~7.

\subsection*{Scope}

Equations \eqref{eq:EOM_linear}--\eqref{eq:EOM_nonlinear} are used as effective,
weak-field relations for galaxy-scale dynamics only.
No claims are made regarding strong-field gravity or precision Solar-System tests.
