%==============================================================================
% Section 4: Minimal Enchan Field Model
%==============================================================================

\chap{Minimal Enchan Field Model}

In this chapter, we formalize the Enchan framework as a scalar field theory. We construct a Lagrangian density that describes how the time-dilation field $S$ propagates through the "rigid" vacuum of the Inception World, driven by primordial fluctuations and stabilized by a potential.

\subsection{Field Variables and Dimensional Analysis (\version)}

To ensure physical consistency with the derivation of $a_0$ (Chapter 5), we explicitly fix the dimensions of the fields and constants. We adopt the standard SI units where the Lagrangian density $\mathcal{L}$ has dimensions of Energy Density $[J/m^3]$.

\begin{itemize}
    \item \textbf{Enchan Field $S(x)$ [Dimensionless]:}
    The order parameter representing the local time dilation. $S=0$ corresponds to the asymptotic vacuum (free fall).

    \item \textbf{Vacuum Stiffness $\sigma_{\text{vac}}$ [Force, N]:}
    A fundamental coupling constant representing the "rigidity" of the spacetime fabric against time dilation. It converts the dimensionless gradient $(\nabla S)^2$ into an energy density.
    
    \item \textbf{Fluctuation Source $F(x)$ [Energy Density, $J/m^3$]:}
    The external driving force from the Upper Universe (e.g., the roughness of the singularity).

    \item \textbf{Stabilizing Potential $V(S)$ [Energy Density, $J/m^3$]:}
    The self-interaction energy of the field, responsible for forming topological defects.
\end{itemize}

\subsection{The Lagrangian Density}

We propose the following effective Lagrangian density for the Enchan field:

\begin{equation}
    \boxed{
    \mathcal{L} \;=\;
    \frac{\sigma_{\text{vac}}}{2} (\partial_\mu S)(\partial^\mu S)
    \;+\; F S
    \;-\; V(S)
    }
    \label{eq:Lagrangian}
\end{equation}

\paragraph{Physical Interpretation of Terms:}
\begin{enumerate}
    \item \textbf{Kinetic Term ($\frac{\sigma_{\text{vac}}}{2} (\partial S)^2$):}
    Represents the elastic energy cost of having "uneven time." Because $\sigma_{\text{vac}}$ is large (spacetime is rigid), gradients in $S$ are energetically costly. This term eventually manifests as the "Geometric Dark Matter" energy density.
    
    \item \textbf{Source Term ($F S$):}
    Represents the coupling to the primordial fluctuations. A non-zero $F$ drives $S$ away from zero, creating initial seeds for structures.
    
    \item \textbf{Potential Term ($-V(S)$):}
    Represents the "Anchoring" energy. As derived in Chapter 5, the requirement for logarithmic defects imposes a specific form on this potential: $V(S) \sim e^{-2S}$.
\end{enumerate}

\subsection{Equation of Motion}

Varying the action associated with Eq.~\eqref{eq:Lagrangian} with respect to $S$ yields the Euler-Lagrange equation:

\begin{equation}
    \partial_\mu \left( \frac{\partial \mathcal{L}}{\partial (\partial_\mu S)} \right) - \frac{\partial \mathcal{L}}{\partial S} = 0
\end{equation}

Substituting our specific Lagrangian:

\begin{equation}
    \sigma_{\text{vac}} \partial_\mu \partial^\mu S \;=\; F \;-\; \frac{dV}{dS}
\end{equation}

Or, using the d'Alembertian operator $\square$:

\begin{equation}
    \boxed{
    \sigma_{\text{vac}} \square S \;=\; F \;-\; V'(S)
    }
    \label{eq:FieldEq}
\end{equation}

This is the fundamental field equation of Enchan \version. It states that:
\begin{quote}
    \textit{"The curvature of time ($\square S$) is driven by the primordial roughness ($F$) and resisted by the anchoring potential ($V'$), scaled by the stiffness of the vacuum ($\sigma_{\text{vac}}$)."}
\end{quote}

\subsection{Static Limit and Dark Matter Halo}

In the static limit (galactic scales) and far from the core (where $F \approx 0$), the equation simplifies to:
\begin{equation}
    \sigma_{\text{vac}} \nabla^2 S \approx -V'(S).
\end{equation}
This is the equation we solved in Chapter 5 to derive the $a_0$ scale. The resulting gradient energy density is:
\begin{equation}
    \rho_{\text{DM}} \approx \frac{\sigma_{\text{vac}}}{2} (\nabla S)^2.
\end{equation}
Since the defect solution gives $|\nabla S| \propto 1/r$, the energy density scales as:
\begin{equation}
    \rho_{\text{DM}} \propto \frac{1}{r^2}.
\end{equation}
This $1/r^2$ profile is exactly what is required to produce flat rotation curves in galaxies, identifying the "Enchan Field Gradient" as the physical substance of Dark Matter halos.

\subsection{Relation to General Relativity (Effective Metric)}

Although we treat $S$ as a scalar field on a background, it effectively modifies the metric experienced by matter. The line element in the presence of an Enchan field can be approximated (in the weak field limit) as:
\begin{equation}
    ds^2 \approx -c^2 e^{-2S} dt^2 + e^{2S} d\mathbf{x}^2.
\end{equation}
This confirms that $S$ acts as a conformal factor or a time-dilation potential. Matter follows the geodesics of this effective metric, which leads to the observed "extra gravity" without requiring additional mass.