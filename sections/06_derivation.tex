%===========================
% Chapter 6: Derivation / Acceleration Scale
%===========================

\chap{Derivation: the acceleration scale}

\subsection{Purpose and scope}
This chapter provides a \textbf{model-facing derivation ledger} for the acceleration scale $a_0$.
The goal is not to claim completion, but to fix:
(i) the assumptions used in the derivation,
(ii) the dimensional bookkeeping, and
(iii) a concrete, testable dependence that can be confronted with public data.

\subsection{Assumptions used in this chapter}
We work under the following controlled assumptions:
\begin{itemize}
\item \textbf{Static limit:} $\partial_t S \simeq 0$ on galactic scales.
\item \textbf{Spherical defect ansatz:} the dominant large-scale defect profile is treated as effectively spherical.
\item \textbf{Anchor condition (ansatz):} the defect core energy density is tied to a baryonic \emph{surface}-density proxy evaluated at a characteristic anchor radius $r=r_c$.
\end{itemize}
These assumptions are explicitly flagged so that later revisions can replace them with a more fundamental construction.

\subsection{Field normalization and dimensions}
We adopt the \version convention that the Enchan field $S$ is \textbf{dimensionless} and is interpreted as an effective time-dilation factor.
The Lagrangian density is written schematically as
\begin{equation}
\mathcal{L} = \frac{\sigma_{\rm vac}}{2}(\partial_\mu S)(\partial^\mu S) - V(S),
\end{equation}
where $\sigma_{\rm vac}$ is a universal stiffness parameter (vacuum ``rigidity'') and $V(S)$ is an effective potential encoding anchoring and defect-core energetics.
This chapter uses $\sigma_{\rm vac}$ and a defect-core energy density scale $V_0$ to define the length and acceleration scales below.

\subsection{Defect length scale}
We introduce a characteristic core/anchor length scale $\ell_c$ via
\begin{equation}
\ell_c \equiv \sqrt{\frac{\sigma_{\rm vac}}{2V_0}},
\label{eq:lc_def}
\end{equation}
where $V_0$ has units of energy density and represents the effective defect-core energy density associated with the anchoring region.
This definition is used as a convenient parameterization of the ``stiffness versus core energy'' balance.

\subsection{Defect profile ansatz}
At radii outside the core, we take a logarithmic defect profile,
\begin{equation}
S(r) \simeq S_0 + \eta_S \ln\!\left(\frac{r}{\ell_c}\right),
\label{eq:S_log_profile}
\end{equation}
where $\eta_S$ is a dimensionless defect-strength parameter and $S_0$ is a gauge constant.
This log profile is the minimal form that yields an asymptotic gradient $|\nabla S|\propto 1/r$.

\subsection{The Surface-Density Anchor: A Pressure-Matching Condition}
\label{sec:anchor_derivation}

We now formulate the local condition that determines the topological defect strength from the baryonic distribution.
A key design choice is to anchor the defect using a \textbf{surface-density} proxy rather than a volume density, thereby avoiding the enormous dynamic range of $\rho_{\rm b}$ across galactic scales and focusing on the relevant dynamical quantity for disk systems.

Dimensional analysis provides a robust guide for this coupling.
We introduce an \emph{effective defect stress scale}, $P_{\rm def}$, defined as:
\begin{equation}
P_{\rm def}(r) \equiv \frac{V_0(r)}{\eta_S}.
\end{equation}
Here, $V_0(r)$ represents the local vacuum energy density scale. We normalize by $\eta_S$ to interpret $P_{\rm def}$ as the core energy density \emph{per unit defect strength}, ensuring that the anchor condition constrains the strength-normalized ``stress budget'' of the vacuum texture.
By construction, since energy density and pressure share the same dimensions ($[M][L]^{-1}[T]^{-2}$), $P_{\rm def}$ naturally represents a stress scale.

On the baryonic side, consider a self-gravitating sheet with surface mass density $\Sigma_{\rm b}$. The characteristic gravitational pressure (or vertical stress) within the sheet scales as $P_{\rm sg} \sim G \Sigma_{\rm b}^2$ (e.g., the midplane vertical stress scaling for a self-gravitating sheet), up to an order-unity geometric factor.

We therefore postulate that the defect's effective pressure scale matches the baryonic gravitational pressure at the anchor radius $r_c$.
The \textit{minimal anchor ansatz} is then the condition of pressure equilibrium:
\begin{equation}
P_{\rm def}(r_c) \simeq \frac{1}{\alpha_c} P_{\rm sg}(r_c)
\quad \Longrightarrow \quad
\frac{V_0(r_c)}{\eta_S} \simeq \frac{1}{\alpha_c} \, G \, \Sigma_{\rm b}(r_c)^2,
\label{eq:V0_anchor}
\end{equation}
where $\alpha_c$ is a dimensionless coupling efficiency factor of order unity.
This relation implies that the vacuum texture is locally "pinned" or regulated by the self-gravity of the baryonic sheet.
Equation (\ref{eq:V0_anchor}) recovers the scaling dependence used in Eq.~(19), but frames it as a leading-order effective field theory (EFT) description consistent with pressure dimensions, rather than an arbitrary choice.

\subsubsection{Observational Proxy for \texorpdfstring{$\Sigma_{\rm b}$}{Sigma\_b}}
For practical confrontation with SPARC data, we treat $\Sigma_{\rm b}(r_c)$ not as a theoretically exact quantity, but as an \textbf{observational proxy} derived from surface photometry.
Assuming the disk dominates the potential at the anchor radius, we adopt:
\begin{equation}
\Sigma_{\rm b}(r_c) \approx \Upsilon_* I_{\rm disk}(r_c) + \Sigma_{\rm gas}(r_c),
\label{eq:sigma_proxy}
\end{equation}
where $\Upsilon_*$ is the stellar mass-to-light ratio and $I_{\rm disk}$ is the observed luminosity profile.

We acknowledge that this proxy carries systematic uncertainties (e.g., from $\Upsilon_*$ variations, geometry, or bulge contamination).
However, the primary goal of the differential prediction test (Test C1) is to verify the \textit{sign and power-law index} of the dependence implied by Eq.~(\ref{eq:V0_anchor})—specifically that the relevant acceleration scale scales with surface density—rather than to strictly constrain the normalization constants.
Any global systematic offsets in the proxy are absorbed into the effective coupling factor $\alpha_c$.

\subsubsection{Environmental suppression in deep potentials (``pinning'')}
\label{subsubsec:pinning_suppression}

Sections~\ref{sec:anchor_derivation}--\ref{eq:sigma_proxy} define a baseline (``free'') acceleration scale,
$a_{0,\mathrm{free}}$, controlled by the local baryonic surface density proxy.
This construction is expected to be most reliable in disk-dominated, low-complexity regions.
However, dense central environments (high surface brightness and deep gravitational wells) introduce
additional structure (bulge/bar geometry, noncircular motions, proxy contamination) that can spoil a
single-parameter surface-density predictor.

To encode this regime dependence while preserving the pressure-matching anchor as the leading-order
term, we introduce a minimal \emph{environmental suppression} of the effective acceleration scale in
deep baryonic potentials. Concretely, we define the \emph{observed / effective} acceleration scale as
\begin{equation}
a_{0,\mathrm{eff}}(r_c) \;\equiv\;
a_{0,\mathrm{free}}(\Sigma_{\rm b}(r_c))\;
\mathcal{S}_\Phi\!\left(|\Phi_{\rm bar}(r_c)|\right),
\label{eq:a0_eff_with_phi_screening}
\end{equation}
where $0<\mathcal{S}_\Phi\le 1$ is a dimensionless suppression factor and
$\Phi_{\rm bar}$ is the Newtonian gravitational potential sourced by baryons.

\paragraph{Potential-depth proxy.}
For observational work, we adopt a practical proxy for the potential depth at the anchor radius.
Since $\Phi$ has the dimensions of velocity squared, a convenient estimator is
\begin{equation}
|\Phi_{\rm bar}(r_c)| \;\approx\; V_{\rm bar}(r_c)^2,
\qquad
V_{\rm bar}^2 \equiv V_{\rm gas}^2
+ \Upsilon_d V_{\rm disk}^2
+ \Upsilon_b V_{\rm bul}^2,
\label{eq:phi_proxy_vbar2}
\end{equation}
using the standard SPARC mass-model decomposition.\footnote{
Up to geometry-dependent $\mathcal{O}(1)$ factors, $V^2$ is the natural local proxy for potential depth.
This choice is intended as an operational definition for reproducible tests, not as a unique theoretical
identification.}

\paragraph{Minimal pinning function.}
The simplest EFT-motivated form that (i) leaves the low-potential regime unchanged and
(ii) suppresses the anomaly in deep baryonic wells is
\begin{equation}
\mathcal{S}_\Phi(|\Phi|)
\;=\;
\left[1+\left(\frac{|\Phi|}{\Phi_c}\right)^{n}\right]^{-1},
\label{eq:phi_screening_function}
\end{equation}
with two \emph{global} (non-galaxy-dependent) parameters: a critical potential scale $\Phi_c$
and an index $n>0$.

In the shallow-potential regime, $|\Phi|\ll\Phi_c$, one has $\mathcal{S}_\Phi\to1$ and
Eq.~(\ref{eq:a0_eff_with_phi_screening}) reduces to the baseline surface-density anchor prediction.
In deep baryonic wells, $|\Phi|\gg\Phi_c$, the suppression becomes strong,
$\mathcal{S}_\Phi\sim(|\Phi|/\Phi_c)^{-n}$, effectively ``pinning'' the defect response and reducing
the impact of the variable-$a_0$ correction.

\paragraph{Interpretation and falsifiability.}
Equation~(\ref{eq:a0_eff_with_phi_screening}) is a \emph{minimal regime extension} rather than a
post-hoc per-galaxy tuning: $\Phi_c$ and $n$ are shared across the full sample and may be constrained
(or excluded) by cross-validation.
Operationally, this term predicts that the surface-density dependence of $a_0$ is most visible in
low-potential, disk-dominated environments, while it is suppressed in high-potential central regions.

\subsubsection{High-acceleration suppression in Solar-System extrapolations}
\label{subsubsec:g_screening}

When extrapolated to high-acceleration environments such as the Solar System,
the quadrature-style closure generically exhibit a residual offset
$\Delta g \sim \mathcal{O}(a_0)$ in the limit $g_N \gg a_0$.
Although negligible in galactic low-acceleration regimes, such residuals can
accumulate into observationally significant effects in precision Solar-System
dynamics (e.g., effective phantom masses growing as $r^2$ or anomalous perihelion
precession).

To ensure consistency of Solar-System extrapolations while preserving the
galaxy-calibrated form of Eq.~(\ref{eq:a0_eff_with_phi_screening}),
we introduce an additional \emph{high-acceleration suppression factor}
that acts only in regimes of large Newtonian acceleration.
We write
\begin{equation}
a_{0,\mathrm{eff}}
\;=\;
a_{0,\mathrm{free}}\;
\mathcal{S}_\Phi(|\Phi|)\;
\mathcal{S}_g(g_N),
\label{eq:a0_eff_with_g_screening}
\end{equation}
where $\mathcal{S}_g$ is defined as
\begin{equation}
\mathcal{S}_g(g_N)
\;=\;
\left[1+\left(\frac{g_N}{g_c}\right)^m\right]^{-1},
\label{eq:g_screening_function}
\end{equation}
with $g_c$ a characteristic acceleration scale and $m>0$ an index controlling
the sharpness of the suppression.

By construction, $\mathcal{S}_g\to1$ for $g_N\ll g_c$, leaving galactic predictions
unchanged, while $\mathcal{S}_g\to0$ for $g_N\gg g_c$, suppressing the residual
high-acceleration response.

\paragraph{Scope and interpretation.}
The high-acceleration suppression term is \emph{not} part of the galaxy calibration
and is therefore excluded from the C1 cross-validation analysis.
It should be regarded as a Solar-System extrapolation safeguard, ensuring that the
theory reduces smoothly to Newtonian dynamics in regimes where absolute acceleration
constraints are strongest.



\subsection{Derived acceleration scale}
We define the effective acceleration scale $a_0$ by combining the defect profile parameter $\eta_S$ with the characteristic length $\ell_c$:
\begin{equation}
a_0 \equiv \frac{c^2\,\eta_S}{\ell_c}
= c^2\,\eta_S \sqrt{\frac{2V_0}{\sigma_{\rm vac}}}.
\label{eq:a0_def}
\end{equation}
Using the anchor scaling in Eq.~\eqref{eq:V0_anchor}, this implies the parametric dependence
\begin{equation}
a_0 \propto \eta_S\,\Sigma_{\rm b}(r_c)\,\sqrt{\frac{G}{\sigma_{\rm vac}}}\,,
\label{eq:a0_scaling}
\end{equation}
i.e.\ $a_0$ is not assumed fundamental, but is controlled by vacuum stiffness and an anchored surface-density proxy (up to dimensionless factors).

\subsection{Testable prediction}
Because Eq.~\eqref{eq:a0_scaling} links $a_0$ to a baryonic surface-density proxy, this framework makes a direct, falsifiable statement:
\begin{quote}
\textbf{Prediction:} the per-galaxy acceleration scale inferred from galaxy-level relations should exhibit a \emph{weak} correlation with surface-brightness indicators (or related $\Sigma_{\rm b}$ proxies), rather than being perfectly universal.
\end{quote}
If a near-universal $a_0$ continues to describe diverse galaxies, this must correspond to either
(i) weak variation of the relevant proxy, or
(ii) self-regulation (e.g.\ compensating variation in $\eta_S$ and/or $\alpha_c$).
Either outcome is informative and can be tested.

\subsection{Connection to the empirical benchmarks}
Sec.~\ref{sec:obs_btfr} and Sec.~\ref{sec:obs_rar} (benchmarks) summarize the observed emergence of an acceleration scale $\sim 10^{-10}\,\mathrm{m/s^2}$ from public data.
This chapter provides a \textbf{theory-side parametrization} of where such a scale could enter the model, and fixes the dependencies that must be checked against the benchmark outputs.

%===========================================================
% \version Addendum: Effective Field Equation & Transition
%===========================================================
\subsection{Effective field equation and a candidate transition function}
\label{sec:transition_function}

This section provides an \emph{effective} (approximate) derivation of the transition
between the low-acceleration and high-acceleration regimes. The goal is to connect
the benchmark closure used in v0.3.x to a field-equation form suitable for \version.

\subsubsection{Quasi-static limit and the source term}
We assume a quasi-static, weak-field regime on galactic scales, and identify the
effective gravitational response with the gradient of the dimensionless time-dilation field:
\begin{equation}
\bm{g}_{\rm tot} \equiv c^2 \nabla S .
\end{equation}
Because matter follows the physical metric that depends on $S$ (time dilation),
the matter sector contributes an effective source for $S$ in the non-relativistic limit.
We therefore adopt the following Poisson-like \emph{effective} equation:
\begin{equation}
\nabla \cdot \left[ \mu\!\left(\frac{g_{\rm tot}}{a_0}\right)\, \bm{g}_{\rm tot} \right]
= 4\pi G \rho_{\rm bar}.
\label{eq:mu_poisson}
\end{equation}
Equation~\eqref{eq:mu_poisson} should be read as an effective macroscopic description
of the vacuum texture response. A microscopic derivation from a specific defect model
is deferred to future work.

\subsubsection{Transition function consistent with the quadrature closure}
In spherical symmetry, Eq.~\eqref{eq:mu_poisson} reduces to the algebraic relation
\begin{equation}
\mu\!\left(\frac{g_{\rm tot}}{a_0}\right)\, g_{\rm tot} = g_{\rm bar},
\qquad
g_{\rm bar}(r)\equiv \frac{G M_{\rm bar}(<r)}{r^2}.
\label{eq:mu_relation}
\end{equation}
The v0.3.x benchmarks use the ``Enchan quadrature'' closure
\begin{equation}
g_{\rm tot}=\sqrt{g_{\rm bar}^2+a_0 g_{\rm bar}}
= g_{\rm bar}\,\nu\!\left(\frac{g_{\rm bar}}{a_0}\right),
\qquad
\nu(y)=\sqrt{1+\frac{1}{y}} .
\label{eq:nu_quadrature}
\end{equation}
We define a candidate $\mu$-function that reproduces this closure:
\begin{equation}
\mu\!\left(x\right)=\frac{g_{\rm bar}}{g_{\rm tot}}
=\frac{\sqrt{1+4x^2}-1}{2x},
\qquad x\equiv \frac{g_{\rm tot}}{a_0}.
\label{eq:mu_quadrature}
\end{equation}
Equations~\eqref{eq:nu_quadrature}--\eqref{eq:mu_quadrature} define a transition function that is \emph{by construction} consistent with the benchmark closure.
The theory-side task for \version+ is to derive (or approximate) this response from an explicit
defect profile and anchoring mechanism, rather than postulating it.

\subsubsection{Why the benchmark closure is numerically innocuous on galaxies but problematic in precision Solar-System tests}
The limits of Eq.~\eqref{eq:nu_quadrature} are:
\begin{itemize}
\item \textbf{Low-acceleration (deep) regime $g_{\rm bar}\ll a_0$:}
\begin{equation}
g_{\rm tot}\simeq \sqrt{a_0 g_{\rm bar}},
\end{equation}
which yields asymptotically flat rotation curves ($v^2\simeq \sqrt{G M_{\rm bar} a_0}$).

\item \textbf{High-acceleration regime $g_{\rm bar}\gg a_0$:}
\begin{equation}
g_{\rm tot}= g_{\rm bar}\sqrt{1+\frac{a_0}{g_{\rm bar}}}
\simeq g_{\rm bar}\left(1+\frac{a_0}{2g_{\rm bar}}\right).
\end{equation}
For Solar-System/terrestrial accelerations ($g_{\rm bar}\gg a_0$), the fractional correction
$\Delta g/g_{\rm bar}\sim a_0/(2g_{\rm bar})$ is extremely small.
However, the quadrature closure leaves a residual \emph{absolute} offset $\Delta g \sim \mathcal{O}(a_0)$,
which can accumulate into measurable signatures in precision Solar-System dynamics.
This motivates treating high-acceleration extrapolations separately via Eq.~(\ref{eq:a0_eff_with_g_screening}).
\end{itemize}

\subsubsection{Connection to the RAR benchmarks}
Using $\nu(y)=\sqrt{1+1/y}$, the closure can be written as
\begin{equation}
g_{\rm tot}=g_{\rm bar}\,\nu\!\left(\frac{g_{\rm bar}}{a_0}\right),
\end{equation}
which is the mapping implemented in the v0.3.x reproducibility scripts. In this sense,
the RAR benchmark can be restated as a \emph{field-equation-compatible} effective law,
pending a microscopic derivation of $\mu$ (or $\nu$) from the Enchan defect model.