%===========================
% Chapter 6: Derivation / Acceleration Scale
%===========================

\chap{Derivation: the acceleration scale}

\subsection{Scope}

This chapter fixes the minimal definitions and parametric dependencies used to
introduce an effective acceleration scale $a_0$ within the field description.

\subsection{Controlled assumptions}

We work under the following assumptions:
\begin{itemize}
\item \textbf{Quasi-static limit:} $\partial_t S \simeq 0$ on galaxy scales.
\item \textbf{Spherical profile (outer region):} the dominant large-radius behavior is
treated as effectively spherical.
\item \textbf{Anchor ansatz:} a defect-core energy-density scale is tied to a baryonic
surface-density proxy evaluated at a characteristic radius $r=r_c$.
\end{itemize}

\subsection{Core length scale}

Let $S$ be dimensionless and consider an effective Lagrangian with a constant kinetic
coefficient $\sigma$ and an effective core energy-density scale $V_0$.
We define a characteristic core/anchor length scale
\begin{equation}
\ell_c \equiv \sqrt{\frac{\sigma}{2V_0}}.
\label{eq:lc_def}
\end{equation}

\subsection{Outer defect profile}

Outside the core region, we adopt a logarithmic profile as a minimal asymptotic form,
\begin{equation}
S(r) \simeq S_0 + \eta_S \ln\!\left(\frac{r}{\ell_c}\right),
\label{eq:S_log_profile}
\end{equation}
where $\eta_S$ is a dimensionless amplitude and $S_0$ is an additive constant.
This implies
\begin{equation}
|\nabla S| = \frac{\eta_S}{r}.
\end{equation}

\subsection{Definition of the acceleration scale}

We define the effective acceleration scale by combining $\eta_S$ and $\ell_c$:
\begin{equation}
a_0 \equiv \frac{c^2\,\eta_S}{\ell_c}
= c^2\,\eta_S \sqrt{\frac{2V_0}{\sigma}}.
\label{eq:a0_def}
\end{equation}

\subsection{Surface-density anchor (minimal form)}

Let $\Sigma_{\rm b}(r_c)$ be a baryonic surface-density proxy evaluated at the
characteristic radius $r_c$.
We postulate a leading-order anchoring relation of the form
\begin{equation}
\frac{V_0(r_c)}{\eta_S} \simeq \frac{1}{\alpha_c}\,G\,\Sigma_{\rm b}(r_c)^2,
\label{eq:V0_anchor}
\end{equation}
where $\alpha_c$ is a dimensionless efficiency parameter.
Together with Eq.~\eqref{eq:a0_def}, this fixes the parametric dependence of $a_0$
on the surface-density proxy up to dimensionless factors.

\subsection{Environmental suppression in deep potentials}

We allow a minimal environmental suppression of the effective acceleration scale in
deep baryonic potentials by defining
\begin{equation}
a_{0,\mathrm{eff}}(r_c) \equiv
a_{0,\mathrm{free}}\!\big(\Sigma_{\rm b}(r_c)\big)\;
\mathcal{S}_\Phi\!\left(|\Phi_{\rm bar}(r_c)|\right),
\label{eq:a0_eff_with_phi_screening}
\end{equation}
with a suppression factor $0<\mathcal{S}_\Phi\le 1$.
Operationally, we use the standard baryonic mass-model proxy
\begin{equation}
|\Phi_{\rm bar}(r_c)| \approx V_{\rm bar}(r_c)^2,
\qquad
V_{\rm bar}^2 \equiv V_{\rm gas}^2 + \Upsilon_d V_{\rm disk}^2 + \Upsilon_b V_{\rm bul}^2,
\label{eq:phi_proxy_vbar2}
\end{equation}
and adopt the minimal two-parameter form
\begin{equation}
\mathcal{S}_\Phi(|\Phi|)
=
\left[1+\left(\frac{|\Phi|}{\Phi_c}\right)^{n}\right]^{-1},
\label{eq:phi_screening_function}
\end{equation}
with global parameters $(\Phi_c, n)$ shared across the sample.

\subsection*{Connection to observational benchmarks}

Chapter~7 confronts the model with observational benchmarks using the dependencies
fixed in this chapter.
