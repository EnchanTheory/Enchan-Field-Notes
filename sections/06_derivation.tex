%===========================
% Chapter 6: Theoretical Derivation (v0.3.0)
%===========================

\chap{Derivation of the acceleration scale}

\subsection*{Purpose and status}
This chapter provides a \textbf{dimensionally consistent} derivation path from the v0.3.0 field definition
to the emergence of an acceleration scale. The point is to make the logic auditable:
\begin{itemize}
\item what follows from the Lagrangian and a chosen static defect profile,
\item what is an \textbf{anchor condition} (phenomenological input) connecting baryons to the defect scale,
\item what becomes a numerical target for the public-data benchmarks (RAR / BTFR / rotation-curve shape).
\end{itemize}

\subsection*{Dimensional bookkeeping (fixed)}
We adopt the v0.3.0 convention:
\begin{itemize}
\item The Enchan field $S$ is \textbf{dimensionless}.
\item It is interpreted as a time-dilation / potential-like quantity, so that in the weak-field,
quasi-static regime we read an effective acceleration from
\begin{equation}
g_S(r) \;\equiv\; c^2\,\left|\nabla S(r)\right| ,
\label{eq:gS_def}
\end{equation}
consistent with the identification $S \sim \Phi/c^2$.
\item The stiffness $\sigma_{\rm vac}$ carries units such that
$\sigma_{\rm vac}(\nabla S)^2$ has units of energy density. A consistent choice is
$[\sigma_{\rm vac}] = {\rm J/m}$ (equivalently N), since $[\nabla S]^2 = 1/{\rm m^2}$.
\end{itemize}

%----------------------------------------------------------
\subsection{Field equation from the v0.3.0 Lagrangian}
\label{sec:derivation_eom}

We start from the Lagrangian density introduced in Chapter 4 (Eq.~\eqref{eq:Lagrangian}),
\begin{equation}
\mathcal{L} \;=\; \frac{\sigma_{\rm vac}}{2}(\partial_\mu S)(\partial^\mu S) \;+\; F S \;-\; V(S).
\end{equation}
The Euler--Lagrange equation gives
\begin{equation}
\sigma_{\rm vac}\,\Box S \;+\; \frac{dV}{dS} \;=\; F .
\label{eq:EL_full}
\end{equation}

For a quasi-static galactic configuration in a locally flat background, we use
$\Box \rightarrow \nabla^2$ and obtain
\begin{equation}
\sigma_{\rm vac}\,\nabla^2 S \;+\; \frac{dV}{dS} \;=\; F.
\label{eq:EL_static}
\end{equation}
In the outer region (outside the dominant baryonic distribution), it is useful to treat $F$ as negligible
and interpret the persistent nontrivial profile as a \textbf{defect/boundary-condition} solution:
\begin{equation}
\sigma_{\rm vac}\,\nabla^2 S \;+\; \frac{dV}{dS} \;\simeq\; 0 \qquad (r \gtrsim \text{baryonic extent}).
\label{eq:outer_eq}
\end{equation}

%----------------------------------------------------------
\subsection{Inverse problem: potential compatible with a logarithmic defect}
\label{sec:derivation_potential}

We now impose the v0.3.0 defect ansatz
\begin{equation}
S(r) \;=\; \eta_S \,\ln\!\left(\frac{r}{r_c}\right),
\label{eq:log_defect}
\end{equation}
where $\eta_S$ is dimensionless and $r_c$ is a core/anchor length scale.

For spherical symmetry,
\begin{equation}
\nabla^2 S \;=\; \frac{1}{r^2}\frac{d}{dr}\left(r^2\frac{dS}{dr}\right).
\end{equation}
Using $dS/dr = \eta_S/r$ from Eq.~\eqref{eq:log_defect}, we obtain
\begin{equation}
\nabla^2 S \;=\; \frac{\eta_S}{r^2}.
\label{eq:laplace_log}
\end{equation}

Requiring Eq.~\eqref{eq:outer_eq} to hold for the profile \eqref{eq:log_defect} implies
\begin{equation}
\frac{dV}{dS} \;=\; -\,\sigma_{\rm vac}\,\frac{\eta_S}{r^2}.
\label{eq:dVdS_rform}
\end{equation}
To rewrite the right-hand side purely as a function of $S$, eliminate $r$ using
$r = r_c\,e^{S/\eta_S}$ from Eq.~\eqref{eq:log_defect}, giving
\begin{equation}
\frac{1}{r^2} \;=\; \frac{1}{r_c^2}\,\exp\!\left(-\frac{2S}{\eta_S}\right).
\end{equation}
Thus Eq.~\eqref{eq:dVdS_rform} becomes
\begin{equation}
\frac{dV}{dS} \;=\; -\,\sigma_{\rm vac}\,\frac{\eta_S}{r_c^2}\,
\exp\!\left(-\frac{2S}{\eta_S}\right).
\end{equation}
Integrating with respect to $S$ yields the required potential form
\begin{equation}
V(S) \;=\; V_0\,\exp\!\left(-\frac{2S}{\eta_S}\right) \;+\; \text{const},
\label{eq:V_exponential}
\end{equation}
with the amplitude fixed by consistency to
\begin{equation}
V_0 \;=\; \frac{\sigma_{\rm vac}\,\eta_S^2}{2\,r_c^2}.
\label{eq:V0_def}
\end{equation}

\noindent
This is the precise sense in which an \textbf{exponential anchor potential} is singled out by the
requirement that a logarithmic defect profile be a static solution.

%----------------------------------------------------------
\subsection{Emergent acceleration and flat rotation curves}
\label{sec:derivation_accel}

From Eq.~\eqref{eq:log_defect}, the gradient is
\begin{equation}
\left|\nabla S\right| \;=\; \left|\frac{dS}{dr}\right| \;=\; \frac{\eta_S}{r}.
\end{equation}
Using the v0.3.0 identification \eqref{eq:gS_def}, the defect contribution to the effective acceleration is
\begin{equation}
g_S(r) \;=\; c^2\,\frac{\eta_S}{r}.
\label{eq:gS_1overr}
\end{equation}
If the outer rotation curve is supported by this effective acceleration (dominant at large $r$),
\begin{equation}
\frac{v_{\rm flat}^2}{r} \;\simeq\; g_S(r)
\quad\Rightarrow\quad
v_{\rm flat}^2 \;\simeq\; c^2\,\eta_S,
\label{eq:vflat_eta}
\end{equation}
i.e. a \textbf{flat} rotation curve with a dimensionless coefficient $\eta_S \sim (v_{\rm flat}/c)^2$.

%----------------------------------------------------------
\subsection{Anchor condition and the appearance of a universal scale}
\label{sec:derivation_anchor}

The defect profile introduces a length scale $r_c$ and an amplitude parameter $\eta_S$.
To connect the field solution to baryons, we introduce an \textbf{anchor condition}.

A minimal and observationally convenient choice is to define $r_c$ as the radius where the
baryonic Newtonian acceleration reaches the characteristic scale:
\begin{equation}
g_{\rm bar}(r_c) \;\equiv\; \frac{G\,M_{\rm b}}{r_c^2} \;\simeq\; a_0,
\label{eq:anchor_rc}
\end{equation}
where $M_{\rm b}$ is the (asymptotic) baryonic mass relevant for the outer dynamics.
This gives
\begin{equation}
r_c \;\simeq\; \sqrt{\frac{G\,M_{\rm b}}{a_0}}.
\label{eq:rc_from_a0}
\end{equation}

Evaluate the defect acceleration \eqref{eq:gS_1overr} at $r=r_c$:
\begin{equation}
g_S(r_c) \;=\; c^2\,\frac{\eta_S}{r_c}.
\end{equation}
Imposing $g_S(r_c)\simeq a_0$ (transition matching at the anchor scale) implies
\begin{equation}
a_0 \;\simeq\; c^2\,\frac{\eta_S}{r_c}.
\label{eq:a0_core_form}
\end{equation}
Combining with Eq.~\eqref{eq:rc_from_a0} yields an explicit baryon--defect relation:
\begin{equation}
\eta_S^2 \;\simeq\; \frac{G\,a_0\,M_{\rm b}}{c^4}.
\label{eq:eta_mass_relation}
\end{equation}

Using Eq.~\eqref{eq:vflat_eta}, we immediately obtain
\begin{equation}
v_{\rm flat}^4 \;\simeq\; G\,a_0\,M_{\rm b},
\label{eq:BTFR_slope4}
\end{equation}
which is the \textbf{BTFR slope-4 form}. In this reading, $a_0$ is the \textbf{universal} scale,
while $\eta_S$ varies from galaxy to galaxy and encodes baryonic mass via the anchor condition.

%----------------------------------------------------------
\subsection{Connection to the quadrature closure used in benchmarks}
\label{sec:derivation_quadrature}

The public-data reproducibility benchmarks use the quadrature closure
\begin{equation}
g_{\rm tot} \;=\; \sqrt{g_{\rm bar}^2 + a_0\,g_{\rm bar}},
\label{eq:quadrature_closure}
\end{equation}
which matches the observed RAR trend across the relevant dynamic range.

Within the present framework, Eq.~\eqref{eq:quadrature_closure} can be read as a \textbf{target closure}
under the identification
\begin{equation}
g_S^2 \;\equiv\; a_0\,g_{\rm bar},
\label{eq:gS_target}
\end{equation}
so that $g_{\rm tot} = \sqrt{g_{\rm bar}^2 + g_S^2}$.
In the deep regime ($g_{\rm bar}\ll a_0$) one then has $g_{\rm tot}\approx \sqrt{a_0 g_{\rm bar}}$,
while in the high-acceleration regime ($g_{\rm bar}\gg a_0$) one recovers $g_{\rm tot}\approx g_{\rm bar}$.

\noindent
Important: in v0.3.0 this chapter does \textbf{not} claim that Eq.~\eqref{eq:quadrature_closure}
has been uniquely derived from microphysics. Rather, the goal is to show that a logarithmic defect profile,
together with a physically interpretable anchoring scale, provides a \textbf{consistent theoretical slot}
in which the empirical closure can arise.

%----------------------------------------------------------
\subsection{Expressing a0 in terms of vacuum stiffness and the anchor potential}
\label{sec:derivation_a0_from_constants}

Equation \eqref{eq:a0_core_form} gives a useful dimensional form
\begin{equation}
a_0 \;\simeq\; \frac{c^2}{r_c}\,\eta_S,
\end{equation}
consistent with the v0.3.0 requirement that $a_0$ has dimensions of acceleration.

Using the exponential potential parameterization \eqref{eq:V_exponential}--\eqref{eq:V0_def},
we may also eliminate $(\eta_S, r_c)$ in favor of the ratio $V_0/\sigma_{\rm vac}$:
\begin{equation}
V_0 \;=\; \frac{\sigma_{\rm vac}\,\eta_S^2}{2\,r_c^2}
\quad\Rightarrow\quad
\frac{\eta_S}{r_c} \;=\; \sqrt{\frac{2V_0}{\sigma_{\rm vac}}}.
\end{equation}
Then Eq.~\eqref{eq:a0_core_form} implies the compact expression
\begin{equation}
\boxed{
a_0 \;\simeq\; c^2\,\sqrt{\frac{2V_0}{\sigma_{\rm vac}}}
}
\label{eq:a0_from_V0_sigma}
\end{equation}
which is dimensionally correct:
$[V_0/\sigma_{\rm vac}] = 1/{\rm m^2}$, so $c^2\sqrt{V_0/\sigma_{\rm vac}}$ is an acceleration.

\subsection*{Interpretation}
Eq.~\eqref{eq:a0_from_V0_sigma} states that, in this construction, the acceleration scale is
set by a \textbf{vacuum stiffness} $\sigma_{\rm vac}$ and the \textbf{anchor potential amplitude} $V_0$,
while the galaxy-to-galaxy diversity enters through the anchored defect parameters $(\eta_S, r_c)$,
linked to baryons via Eq.~\eqref{eq:eta_mass_relation}.

%----------------------------------------------------------
\subsection{Summary of assumptions and what remains to be completed}
\label{sec:derivation_status}

\subsection*{What is shown here}
\begin{itemize}
\item A logarithmic defect profile \eqref{eq:log_defect} implies a $1/r$ gradient and thus a flat-curve acceleration \eqref{eq:gS_1overr}.
\item Requiring the defect to satisfy the static field equation outside sources selects an exponential potential \eqref{eq:V_exponential}.
\item Introducing an anchor scale $r_c$ where $g_{\rm bar}(r_c)\sim a_0$ yields BTFR slope-4 in the form \eqref{eq:BTFR_slope4}.
\item The universal acceleration scale can be expressed as \eqref{eq:a0_from_V0_sigma}.
\end{itemize}

\subsection*{What is not yet shown (explicit)}
\begin{itemize}
\item A unique microphysical derivation of the anchor condition \eqref{eq:anchor_rc} and the quadrature closure \eqref{eq:quadrature_closure}.
\item A full relativistic embedding and stress-energy accounting that demonstrates consistency in all regimes.
\end{itemize}

\noindent
Accordingly, v0.3.0 treats this chapter as a \textbf{theoretical consistency bridge}:
it aligns the field definitions with the empirically reproduced scale $a_0$, and it cleanly isolates
the remaining theoretical tasks for an Enchan-derived forward model.
