%===========================
% Chapter 6: Derivation / Acceleration Scale
%===========================

\chap{Derivation: the acceleration scale}

\subsection{Purpose and scope}
This chapter provides a \textbf{model-facing derivation ledger} for the acceleration scale $a_0$.
The goal is not to claim completion, but to fix:
(i) the assumptions used in the derivation,
(ii) the dimensional bookkeeping, and
(iii) a concrete, testable dependence that can be confronted with public data.

\subsection{Assumptions used in this chapter}
We work under the following controlled assumptions:
\begin{itemize}
\item \textbf{Static limit:} $\partial_t S \simeq 0$ on galactic scales.
\item \textbf{Spherical defect ansatz:} the dominant large-scale defect profile is treated as effectively spherical.
\item \textbf{Anchor condition (ansatz):} the defect core energy density is tied to a baryonic \emph{surface}-density proxy evaluated at a characteristic anchor radius $r=r_c$.
\end{itemize}
These assumptions are explicitly flagged so that later revisions can replace them with a more fundamental construction.

\subsection{Field normalization and dimensions}
We adopt the \version convention that the Enchan field $S$ is \textbf{dimensionless} and is interpreted as an effective time-dilation factor.
The Lagrangian density is written schematically as
\begin{equation}
\mathcal{L} = \frac{\sigma_{\rm vac}}{2}(\partial_\mu S)(\partial^\mu S) - V(S),
\end{equation}
where $\sigma_{\rm vac}$ is a universal stiffness parameter (vacuum ``rigidity'') and $V(S)$ is an effective potential encoding anchoring and defect-core energetics.
This chapter uses $\sigma_{\rm vac}$ and a defect-core energy density scale $V_0$ to define the length and acceleration scales below.

\subsection{Defect length scale}
We introduce a characteristic core/anchor length scale $\ell_c$ via
\begin{equation}
\ell_c \equiv \sqrt{\frac{\sigma_{\rm vac}}{2V_0}},
\label{eq:lc_def}
\end{equation}
where $V_0$ has units of energy density and represents the effective defect-core energy density associated with the anchoring region.
This definition is used as a convenient parameterization of the ``stiffness versus core energy'' balance.

\subsection{Defect profile ansatz}
At radii outside the core, we take a logarithmic defect profile,
\begin{equation}
S(r) \simeq S_0 + \eta_S \ln\!\left(\frac{r}{\ell_c}\right),
\label{eq:S_log_profile}
\end{equation}
where $\eta_S$ is a dimensionless defect-strength parameter and $S_0$ is a gauge constant.
This log profile is the minimal form that yields an asymptotic gradient $|\nabla S|\propto 1/r$.

\subsection{Surface-density anchor ansatz}
A key design choice is to anchor the defect using a \textbf{surface-density} proxy rather than a volume density, to avoid the enormous galaxy-to-galaxy dynamic range of $\rho_b$.
We therefore introduce an \emph{effective baryonic surface mass density} $\Sigma_{\rm b}(r_c)$ at an anchor radius $r=r_c$, and postulate the scaling
\begin{equation}
V_0 \simeq \frac{G\,\Sigma_{\rm b}(r_c)^2}{\alpha_c},
\label{eq:V0_anchor}
\end{equation}
where $\alpha_c$ is a dimensionless geometric factor.
This should be read as a \textbf{minimal ansatz}: it is the simplest local energetic scaling constructed from $\Sigma_{\rm b}$ and $G$.

\subsubsection{Observational proxy definition}
For practical confrontation with data, $\Sigma_{\rm b}(r_c)$ is treated as an \textbf{observational proxy} derived from public surface-brightness indicators in SPARC products (e.g.\ disk surface-brightness columns in the mass-model tables, or equivalent catalog proxies).
The purpose here is to fix the dependence and its sign/strength, not to claim a unique mapping from photometry to $\Sigma_{\rm b}$.

\subsection{Derived acceleration scale}
We define the effective acceleration scale $a_0$ by combining the defect profile parameter $\eta_S$ with the characteristic length $\ell_c$:
\begin{equation}
a_0 \equiv \frac{c^2\,\eta_S}{\ell_c}
= c^2\,\eta_S \sqrt{\frac{2V_0}{\sigma_{\rm vac}}}.
\label{eq:a0_def}
\end{equation}
Using the anchor scaling in Eq.~\eqref{eq:V0_anchor}, this implies the parametric dependence
\begin{equation}
a_0 \propto \eta_S\,\Sigma_{\rm b}(r_c)\,\sqrt{\frac{G}{\sigma_{\rm vac}}}\,,
\label{eq:a0_scaling}
\end{equation}
i.e.\ $a_0$ is not assumed fundamental, but is controlled by vacuum stiffness and an anchored surface-density proxy (up to dimensionless factors).

\subsection{Testable prediction}
Because Eq.~\eqref{eq:a0_scaling} links $a_0$ to a baryonic surface-density proxy, this framework makes a direct, falsifiable statement:
\begin{quote}
\textbf{Prediction:} the per-galaxy acceleration scale inferred from galaxy-level relations should exhibit a \emph{weak} correlation with surface-brightness indicators (or related $\Sigma_{\rm b}$ proxies), rather than being perfectly universal.
\end{quote}
If a near-universal $a_0$ continues to describe diverse galaxies, this must correspond to either
(i) weak variation of the relevant proxy, or
(ii) self-regulation (e.g.\ compensating variation in $\eta_S$ and/or $\alpha_c$).
Either outcome is informative and can be tested.

\subsection{Connection to the empirical benchmarks}
Sec.~\ref{sec:obs_btfr} and Sec.~\ref{sec:obs_rar} (benchmarks) summarize the observed emergence of an acceleration scale $\sim 10^{-10}\,\mathrm{m/s^2}$ from public data.
This chapter provides a \textbf{theory-side parametrization} of where such a scale could enter the model, and fixes the dependencies that must be checked against the benchmark outputs.

%===========================================================
% \version Addendum: Effective Field Equation & Transition
%===========================================================
\subsection{Effective field equation and a candidate transition function}
\label{sec:transition_function}

This section provides an \emph{effective} (approximate) derivation of the transition
between the low-acceleration and high-acceleration regimes. The goal is to connect
the benchmark closure used in v0.3.x to a field-equation form suitable for \version.

\subsubsection{Quasi-static limit and the source term}
We assume a quasi-static, weak-field regime on galactic scales, and identify the
effective gravitational response with the gradient of the dimensionless time-dilation field:
\begin{equation}
\bm{g}_{\rm tot} \equiv c^2 \nabla S .
\end{equation}
Because matter follows the physical metric that depends on $S$ (time dilation),
the matter sector contributes an effective source for $S$ in the non-relativistic limit.
We therefore adopt the following Poisson-like \emph{effective} equation:
\begin{equation}
\nabla \cdot \left[ \mu\!\left(\frac{g_{\rm tot}}{a_0}\right)\, \bm{g}_{\rm tot} \right]
= 4\pi G \rho_{\rm bar}.
\label{eq:mu_poisson}
\end{equation}
Equation~\eqref{eq:mu_poisson} should be read as an effective macroscopic description
of the vacuum texture response. A microscopic derivation from a specific defect model
is deferred to future work.

\subsubsection{Transition function consistent with the quadrature closure}
In spherical symmetry, Eq.~\eqref{eq:mu_poisson} reduces to the algebraic relation
\begin{equation}
\mu\!\left(\frac{g_{\rm tot}}{a_0}\right)\, g_{\rm tot} = g_{\rm bar},
\qquad
g_{\rm bar}(r)\equiv \frac{G M_{\rm bar}(<r)}{r^2}.
\label{eq:mu_relation}
\end{equation}
The v0.3.x benchmarks use the ``Enchan quadrature'' closure
\begin{equation}
g_{\rm tot}=\sqrt{g_{\rm bar}^2+a_0 g_{\rm bar}}
= g_{\rm bar}\,\nu\!\left(\frac{g_{\rm bar}}{a_0}\right),
\qquad
\nu(y)=\sqrt{1+\frac{1}{y}} .
\label{eq:nu_quadrature}
\end{equation}
We define a candidate $\mu$-function that reproduces this closure:
\begin{equation}
\mu\!\left(x\right)=\frac{g_{\rm bar}}{g_{\rm tot}}
=\frac{\sqrt{1+4x^2}-1}{2x},
\qquad x\equiv \frac{g_{\rm tot}}{a_0}.
\label{eq:mu_quadrature}
\end{equation}
Equations~\eqref{eq:nu_quadrature}--\eqref{eq:mu_quadrature} define a transition function that is \emph{by construction} consistent with the benchmark closure.
The theory-side task for \version+ is to derive (or approximate) this response from an explicit
defect profile and anchoring mechanism, rather than postulating it.

\subsubsection{Why the modification ``hides'' in strong-gravity environments}
The limits of Eq.~\eqref{eq:nu_quadrature} are:
\begin{itemize}
\item \textbf{Low-acceleration (deep) regime $g_{\rm bar}\ll a_0$:}
\begin{equation}
g_{\rm tot}\simeq \sqrt{a_0 g_{\rm bar}},
\end{equation}
which yields asymptotically flat rotation curves ($v^2\simeq \sqrt{G M_{\rm bar} a_0}$).

\item \textbf{High-acceleration regime $g_{\rm bar}\gg a_0$:}
\begin{equation}
g_{\rm tot}= g_{\rm bar}\sqrt{1+\frac{a_0}{g_{\rm bar}}}
\simeq g_{\rm bar}\left(1+\frac{a_0}{2g_{\rm bar}}\right).
\end{equation}
For Solar-System/terrestrial accelerations ($g_{\rm bar}\gg a_0$), the fractional correction $\Delta g/g_{\rm bar}\sim a_0/(2g_{\rm bar})$ becomes negligibly small, making the deviation effectively unobservable in practice.
\end{itemize}

\subsubsection{Connection to the RAR benchmarks}
Using $\nu(y)=\sqrt{1+1/y}$, the closure can be written as
\begin{equation}
g_{\rm tot}=g_{\rm bar}\,\nu\!\left(\frac{g_{\rm bar}}{a_0}\right),
\end{equation}
which is the mapping implemented in the v0.3.x reproducibility scripts. In this sense,
the RAR benchmark can be restated as a \emph{field-equation-compatible} effective law,
pending a microscopic derivation of $\mu$ (or $\nu$) from the Enchan defect model.