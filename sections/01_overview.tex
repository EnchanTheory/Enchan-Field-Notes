%==============================================================================
% Section 1: Overview
%==============================================================================

\chap{Overview}

These notes present a compact effective description for organizing
galaxy-scale acceleration regularities in rotationally supported systems.
We introduce a dimensionless scalar field $S(x)$ and work in the weak-field regime
where the relevant observables are the baryonic distribution, rotation curves,
and derived radial accelerations.

\subsection{Minimal definitions}

We treat $S$ as a dimensionless proxy field whose spatial gradients encode an
effective gravitational response in the regime of interest. In the notation of
this document, the weak-field response is summarized by
\begin{equation}
g_{\rm tot} \equiv c^2 \nabla S,
\end{equation}
to be interpreted as an effective relation defining how $S$ parametrizes the
acceleration field used in the phenomenological analysis.
No statement is made here about strong-field gravity or a fundamental completion.

\subsection{Working assumptions (phenomenological)}

The framework uses two working assumptions that are kept explicit throughout:

\begin{enumerate}
    \item \textbf{Anchor ansatz.}
    The baryonic distribution provides an anchoring structure that selects
    characteristic scales for spatial variations of $S$ in galaxies.

    \item \textbf{Environmental modulation.}
    In deep baryonic potentials, the effective acceleration scale can be
    suppressed by an environmental proxy, enabling consistent fits across
    a heterogeneous galaxy sample.
\end{enumerate}

These assumptions are treated as phenomenological inputs whose role is to
organize the data in a controlled way.

\subsection{Empirical targets}

The empirical targets used for calibration and verification are:
(i) the Radial Acceleration Relation (RAR),
(ii) the Baryonic Tully--Fisher Relation (BTFR),
and (iii) rotation-curve shape predictions under fixed-parameter rules.
All comparisons are restricted to the weak-field, galaxy-scale domain.

\subsection{Document structure}

\begin{itemize}
    \item \textbf{Chapter 2} introduces the minimal conceptual setup and notation.
    \item \textbf{Chapter 3} summarizes kinematic relations for rotation curves.
    \item \textbf{Chapter 4} states the field model and the effective equation used.
    \item \textbf{Chapter 5} discusses asymptotic behavior relevant for galactic dynamics.
    \item \textbf{Chapter 6} presents the derivations used to connect the model to an
          effective acceleration scale.
    \item \textbf{Chapter 7} confronts the framework with observational benchmarks.
\end{itemize}

\subsection*{Scope}

This document is limited to a weak-field, galaxy-scale effective description.
It does not address precision Solar-System tests, strong-field gravity,
or early-universe cosmology.
