%==============================================================================
% Section 3: Rotation and Kinematic Context
%==============================================================================

\chap{Rotation and Kinematic Context}

This chapter summarizes the role of rotation in the observational and theoretical
context relevant for galaxy dynamics. Rotation curves provide direct access to the
radial acceleration field inferred from circular motion, and therefore serve as a
primary empirical interface for the effective description developed in later chapters.

\subsection{Rotation curves as acceleration data}

For an axisymmetric system with an observed circular velocity profile $v(r)$,
the centripetal acceleration inferred from kinematics is
\begin{equation}
g_{\rm obs}(r) \equiv \frac{v^2(r)}{r}.
\end{equation}
The baryonic contribution is computed from the luminous components under standard
mass-to-light and geometric assumptions, yielding a baryonic acceleration proxy
$g_{\rm bar}(r)$.
The empirical relations discussed in these notes are formulated in terms of the
pair $(g_{\rm obs}, g_{\rm bar})$ across radii and across galaxies.

\subsection{Relativistic rotation effects (context only)}

In General Relativity, rotating mass distributions admit frame-dependent effects
associated with the gravitomagnetic sector of the metric (often referred to as
frame dragging in the weak-field limit).
In these notes, such effects are mentioned only as contextual motivation for why
rotation can be a relevant structural feature of gravitational systems.
No strong-field modeling or device-level interpretation is assumed or required.

\subsection{Connection to the effective description}

The scalar-field parametrization introduced in Chapters 2 and 4 is connected to
galaxy kinematics through the effective acceleration field.
In particular, the working definition
\begin{equation}
g_{\rm tot} \equiv c^2 \nabla S
\end{equation}
is used to map the scalar field variable to an acceleration response in the weak-field,
quasi-static regime.
The remainder of this document focuses on reproducing the observed regularities
in the kinematic data using baryonic inputs and an effective field equation.

\subsection*{Scope}

This chapter provides kinematic definitions and limited theoretical context.
It introduces no new assumptions beyond those stated in Chapters 1--2.
