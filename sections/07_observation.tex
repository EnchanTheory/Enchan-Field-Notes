%===========================
% Chapter 7: Observation / Benchmarks
%===========================

\chap{Observational anchors and benchmarks}

\subsection*{Scope}

This chapter defines a minimal set of observational benchmarks that the effective
description must reproduce in the weak-field, galaxy-scale regime.
Only definitions are fixed here; no implementation details, code structure,
or procedural roadmap is included.

\subsection{Benchmark A: RAR / MDAR (multi-point relation)}
\label{sec:obs_rar}

For each radial point in each galaxy, define the observed centripetal acceleration
\begin{equation}
g_{\rm obs}(r) \equiv \frac{V_{\rm obs}(r)^2}{r},
\end{equation}
and the baryonic acceleration proxy from the standard mass-model decomposition
\begin{equation}
g_{\rm bar}(r) \equiv \frac{V_{\rm gas}(r)^2 + \Upsilon_d V_{\rm disk}(r)^2 + \Upsilon_b V_{\rm bul}(r)^2}{r},
\end{equation}
with consistent unit conversion.

The benchmark statement is the existence of a tight empirical mapping between
$g_{\rm bar}$ and $g_{\rm obs}$ across many radii and many galaxies, commonly
summarized by a one-parameter interpolation curve involving an acceleration scale
$a_0$.

\subsection{Benchmark B: BTFR (one galaxy = one point)}
\label{sec:obs_btfr}

Define a galaxy-level relation between baryonic mass $M_{\rm b}$ (stars + gas) and a
characteristic outer/flat rotation speed $V_{\rm f}$.
In log form,
\begin{equation}
x \equiv \log_{10}(V_{\rm f}/\mathrm{km\,s^{-1}}), \qquad
y \equiv \log_{10}(M_{\rm b}/M_\odot),
\end{equation}
and the benchmark is that a near-linear relation holds:
\begin{equation}
y = a + b\,x,
\end{equation}
with small intrinsic scatter.

A commonly used acceleration-scale diagnostic associated with BTFR is
\begin{equation}
a_{0,{\rm BTFR}} \equiv \frac{V_{\rm f}^4}{G\,M_{\rm b}},
\end{equation}
which provides a per-galaxy summary scale for comparison with the RAR-scale
normalization.

\subsection{Benchmark C: Rotation-curve prediction (shape test)}
\label{sec:obs_rcpred}

Using the baryonic mass-model components at each radius, compute $g_{\rm bar}(r)$.
Apply a fixed (globally shared) mapping to obtain a predicted response
$g_{\rm pred}(r)$, and define
\begin{equation}
V_{\rm pred}(r) = \sqrt{g_{\rm pred}(r)\,r}.
\end{equation}

The benchmark requirement is that a single fixed rule, without per-galaxy tuning,
produces nontrivial rotation-curve shapes across a large heterogeneous sample, and
that deviations exhibit interpretable systematics.

\subsection{Interpretation}

These benchmarks do not by themselves select a unique fundamental theory.
They serve as externally defined targets: an Enchan-derived forward model must
reproduce (A)--(C) within a unified mechanism and a consistent parameterization.
