%===========================
% Chapter 7: Observation / Reproducible Benchmarks
%===========================

\chap{Observational anchors and reproducible benchmarks}

\subsection*{Purpose of this chapter}
This chapter fixes a small set of \textbf{public-data benchmarks} that the Enchan model
must ultimately reproduce \emph{from its own field equation}. The intent is practical:

\begin{itemize}
\item Provide \textbf{externally checkable targets} (data, definitions, numbers, artifacts).
\item Keep claims \textbf{strictly separated}:
  \begin{itemize}
  \item \textbf{Empirical / reproduced}: verified directly from public data with deterministic code.
  \item \textbf{Baseline mapping}: an empirical closure used only as a compact benchmark interface.
  \item \textbf{Enchan-derived (future)}: what must be derived from the Enchan field equation.
  \end{itemize}
\end{itemize}

\subsection*{Status note (baseline vs Enchan-derived)}
\begin{quote}
\textbf{Status (baseline / not yet Enchan-derived):}
The benchmarks below are computed from public datasets using standard, published
RAR/BTFR-style definitions. Some steps use an \emph{empirical closure} as a baseline.
They are kept here as reproducible regression tests and as targets for a future
Enchan-derived forward model.
\end{quote}

\subsection*{Data sources (public)}
We use the following public sources:
\begin{itemize}
\item \textbf{SPARC} mass-model rotation-curve decomposition archive
(\texttt{Rotmod\_LTG.zip}) and related tables.\\
\url{https://astroweb.case.edu/SPARC/}
\item \textbf{SPARC BTFR table} distributed as a CDS-style fixed-width file
(e.g. \texttt{BTFR\_Lelli2019.mrt}).\\
\url{https://astroweb.case.edu/SPARC/}
\end{itemize}

\noindent
Repository policy: large upstream data files (e.g. \texttt{Rotmod\_LTG.zip}) are not committed.
Each script records the \textbf{SHA256} hash of the local input file to make results comparable.

\subsection*{Reproducibility artifacts (this repository)}
The following reproducibility tools (Python scripts + README) are intended to be runnable
on a standard local environment:
\begin{itemize}
\item \textbf{RAR benchmark (multi-point):} \texttt{Enchan\_RAR\_Test\_Report\_v0\_1/}
\item \textbf{BTFR benchmark (one point per galaxy):} \texttt{Enchan\_BTFR\_Test\_Report\_v0\_1/}
\item \textbf{Rotation-curve prediction benchmark (shape test):}
\texttt{Enchan\_SPARC\_Rotation\_Curve\_Prediction\_Report\_v0\_1/}
\end{itemize}

\bigskip

%----------------------------------------------------------
\subsection{Benchmark A: RAR / MDAR (multi-point relation)}
\label{sec:obs_rar}

\subsubsection*{Definition}
For each radial point in each galaxy, we form:
\begin{align}
g_{\rm obs}(r) &= \frac{V_{\rm obs}(r)^2}{r},\\
g_{\rm bar}(r) &= \frac{V_{\rm gas}(r)^2 + Y_d\,V_{\rm disk}(r)^2 + Y_b\,V_{\rm bul}(r)^2}{r},
\end{align}
with unit conversion to $\mathrm{m/s^2}$.

\subsubsection*{Baseline mapping used for summarizing the trend}
A widely used one-parameter empirical curve is used as a compact summary:
\begin{equation}
g_{\rm obs}(g_{\rm bar}; a_0) =
\frac{g_{\rm bar}}{1 - \exp\!\left(-\sqrt{g_{\rm bar}/a_0}\right)}.
\end{equation}
This is treated as a \textbf{benchmark closure} (not yet derived from the Enchan field equation).

\subsubsection*{Reproduced result (public SPARC data)}
Using \texttt{Rotmod\_LTG.zip} (175 galaxies; 3391 points; SB-clean subset defined by
\texttt{SBdisk>0} giving 3111 points), the relation is reproduced with small scatter.
A representative reproduced summary is:

\begin{itemize}
\item Baseline mass-to-light setting: $Y_d=0.50$, $Y_b=0.70$\\
best-fit $a_0 \approx 1.39\times 10^{-10}\,\mathrm{m/s^2}$, RMS scatter $\approx 0.213$ dex.
\item SB-trend-minimized setting (scan in $Y_d$ at fixed $Y_b=0.70$): $Y_d \approx 0.60$\\
best-fit $a_0 \approx 1.12\times 10^{-10}\,\mathrm{m/s^2}$, RMS scatter $\approx 0.212$ dex,
with residual correlation vs SBdisk reduced to near zero (galaxy-median statistic).
\end{itemize}

\subsubsection*{Why this benchmark matters for Enchan}
Empirical content (reproduced): \textbf{across many galaxies and radii, $g_{\rm bar}$ nearly fixes $g_{\rm obs}$.}
Any successful Enchan-derived model must reproduce:
\begin{itemize}
\item the existence of a tight multi-point mapping,
\item the appearance of an acceleration scale $a_0 \sim 10^{-10}\,\mathrm{m/s^2}$,
\item the small scatter and its residual systematics (including SB-related trends under specific choices of $Y_d$).
\end{itemize}

\bigskip

%----------------------------------------------------------
\subsection{Benchmark B: BTFR (one galaxy = one point)}
\label{sec:obs_btfr}

\subsubsection*{Definition}
BTFR is treated as a galaxy-level mapping between baryonic mass $M_{\rm b}$
(stars + gas) and a characteristic outer/flat rotation speed $V_{\rm f}$.
From a public SPARC BTFR table (e.g. \texttt{BTFR\_Lelli2019.mrt}) we extract:
\begin{equation}
x \equiv \log_{10}(V_{\rm f}/\mathrm{km\,s^{-1}}), \qquad
y \equiv \log_{10}(M_{\rm b}/M_\odot),
\end{equation}
and fit a log--log line:
\begin{equation}
y = a + b\,x.
\end{equation}

\subsubsection*{Reproduced result (public table)}
After basic finite-value cuts and requiring $V_{\rm f}>0$, a representative reproduction yields
$N=123$ galaxies with:
\begin{equation}
a \approx 2.188,\qquad b \approx 3.748,\qquad \mathrm{RMS} \approx 0.235~\mathrm{dex}
\end{equation}
(scatter in $y=\log_{10} M_{\rm b}$ around the best-fit line).

\subsubsection*{Acceleration-scale diagnostic from BTFR}
A commonly used diagnostic is the per-galaxy estimate
\begin{equation}
a_{0,{\rm BTFR}} \equiv \frac{V_{\rm f}^4}{G\,M_{\rm b}},
\end{equation}
which has units of acceleration and is expected to be $\mathcal{O}(10^{-10}\,\mathrm{m/s^2})$
when the BTFR is close to slope $\sim 4$.

\noindent
In our reproducibility script, the distribution of $a_{0,{\rm BTFR}}$ is treated as a
\textbf{numerical target} (baseline / not yet Enchan-derived). A representative run yields:
\begin{itemize}
\item median $a_{0,{\rm BTFR}} \approx 1.53\times 10^{-10}\,\mathrm{m/s^2}$,
\item 16--84\% range $\approx (0.90$--$2.65)\times 10^{-10}\,\mathrm{m/s^2}$.
\end{itemize}

\subsubsection*{Why this benchmark matters for Enchan}
Enchan must ultimately explain \emph{why} a stable acceleration scale appears in galaxy-level
data summaries, and how it relates (or does not relate) to the multi-point RAR scale.

\bigskip

%----------------------------------------------------------
\subsection{Benchmark C: Rotation-curve prediction (shape test)}
\label{sec:obs_rcpred}

\subsubsection*{Definition (fixed-parameter prediction)}
Using the SPARC mass-model components at each radius, we compute $g_{\rm bar}(r)$ and apply
a fixed mapping to obtain a predicted response $g_{\rm pred}(r)$, then predict:
\begin{equation}
V_{\rm pred}(r) = \sqrt{g_{\rm pred}(r)\,r}.
\end{equation}
In the baseline benchmark, parameters are fixed globally (no per-galaxy tuning):
\begin{equation}
(Y_d, Y_b) = (0.60, 0.70),\qquad a_0 = 1.12\times 10^{-10}\,\mathrm{m/s^2}.
\end{equation}

\subsubsection*{Reproduced result (public SPARC data)}
A representative reproduction over the full archive (175 galaxies; 3391 points) yields:
\begin{itemize}
\item global RMS residual in $\log_{10} g$: $\approx 0.208$ dex,
\item global RMS fractional velocity error: $\approx 0.358$,
\item per-galaxy reduced $\chi^2$ values (approximate; using only velocity errors) typically above unity.
\end{itemize}

\subsubsection*{Interpretation (what it does and does not claim)}
This benchmark is \textbf{not} a proof of any specific theory. It is a \textbf{stress test}:
a single fixed mapping produces nontrivial, sample-wide curve shapes without tuning.
A future Enchan-derived model must reproduce at least comparable performance, and must also
explain where and why failures occur.

\bigskip

%----------------------------------------------------------
\subsection{What these benchmarks establish (and what they do not)}
\label{sec:obs_limits}

\subsubsection*{Established here (reproduced, empirical)}
\begin{itemize}
\item A tight multi-point RAR/MDAR is strongly present in public SPARC mass-model data.
\item A tight galaxy-level BTFR is strongly present in a public SPARC BTFR table.
\item A fixed-parameter mapping can generate nontrivial rotation-curve predictions at scale.
\end{itemize}

\subsubsection*{Not established here (explicit non-claims)}
\begin{itemize}
\item These results do \textbf{not} falsify particle dark matter.
\item These results do \textbf{not} (yet) derive the closure or the scale $a_0$ from the Enchan field equation.
\item These results do \textbf{not} replace a forward model: they define the targets a forward model must hit.
\end{itemize}

\bigskip

%----------------------------------------------------------
\subsection{Model-facing requirements (for the next iteration)}
\label{sec:obs_requirements}

To claim an \emph{Enchan-derived} explanation, the next step is to replace baseline closures with a
forward prediction from the Enchan field equation, and demonstrate:

\begin{itemize}
\item (i) a derived mapping from baryonic configuration to an effective gravitational response,
\item (ii) an emergent acceleration scale matching $a_0 \sim 10^{-10}\,\mathrm{m/s^2}$,
\item (iii) agreement with the three benchmarks above using the same core mechanism.
\end{itemize}

\noindent
Until (i)--(iii) are complete, the benchmarks in this chapter should be read as
\textbf{targets and regression tests}, not as a completed theory proof.
