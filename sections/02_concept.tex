%==============================================================================
% Section 2: Conceptual Setup and Effective Variables
%==============================================================================

\chap{Conceptual Setup and Effective Variables}

This chapter introduces the minimal conceptual setup used throughout these notes.
We work with a dimensionless scalar field $S(x)$ as an effective proxy variable
in the weak-field regime and define the derived quantities used in later chapters.

\subsection{Field variable}

The scalar field $S(x)$ is taken to be dimensionless.
In the intended regime of applicability, spatial gradients of $S$ are used to
parametrize an effective gravitational response, summarized by
\begin{equation}
g_{\rm tot} \equiv c^2 \nabla S .
\end{equation}
This relation is used as a working definition for the phenomenological mapping
between the field variable and the inferred acceleration field in galaxies.

\subsection{Derived quantities}

Two derived quantities are used repeatedly:

\begin{enumerate}
    \item \textbf{Gradient magnitude.}
    We denote the magnitude of the spatial gradient by
    \begin{equation}
        D(x) \equiv \|\nabla S\| .
    \end{equation}
    In the effective description, $D$ serves as a compact measure of spatial
    variation in $S$.

    \item \textbf{Kinetic invariant.}
    For covariant expressions we define
    \begin{equation}
        \mathcal{X} \equiv \frac{\nabla_\mu S\,\nabla^\mu S}{a_0^2},
    \end{equation}
    where $a_0$ is the characteristic acceleration scale introduced later.
    The function $\mu(\mathcal{X})$ used in the field equation is chosen so that
    the quasi-static limit reproduces the empirical interpolation employed in
    the observational benchmarks.
\end{enumerate}

\subsection{Effective source structure}

The galaxy-scale description uses baryonic structure as the primary source input.
We denote the baryonic mass density by $\rho_{\rm b}$ and adopt an anchoring ansatz
in which baryonic distributions select characteristic scales for spatial variations
of $S$.
Environmental modulation is incorporated through an effective suppression of the
acceleration scale in deep baryonic potentials, using a potential-depth proxy.

\subsection{Notation and conventions}

Unless stated otherwise, we work in the weak-field, nonrelativistic limit when
connecting the field model to observational quantities.
The field $S$ is treated as an effective variable; no assumptions are made here
about a unique fundamental completion beyond the regime studied in these notes.
