%==============================================================================
% Section 5: Asymptotic Structure and Galactic Dynamics
%==============================================================================

\chap{Asymptotic Structure and Galactic Dynamics}

In this chapter, we explore the physical consequences of the Enchan field equations at galactic scales. We show that the "Inception" vacuum naturally supports long-range topological defects, which manifest observationally as "Dark Matter Halos" with a density profile falling as $1/r^2$, leading inevitably to flat rotation curves.

\subsection{The Static Limit}

For a galaxy in equilibrium, we consider the static limit of the field equation derived in Chapter 4:
\begin{equation}
    \sigma_{\text{vac}} \nabla^2 S \approx -V'(S).
\end{equation}
We are interested in the behavior of the field far from the galactic center ($r \gg r_c$), where the direct influence of baryonic matter is negligible, and the dynamics are dominated by the self-interaction of the field (the vacuum structure).

\subsection{The Logarithmic Wrinkle (Topological Defect)}

As derived in Chapter 6, the stability condition for a topological defect in this framework requires the field to behave logarithmically at large distances:
\begin{equation}
    S(r) \sim \eta_S \ln\left(\frac{r}{r_c}\right).
\end{equation}

\paragraph{Physical Interpretation:}
Standard Newtonian gravity relies on the Poisson equation $\nabla^2 \Phi = 4\pi G \rho$, where a point source creates a potential $\Phi \sim 1/r$ and a force $g \sim 1/r^2$.
However, the Enchan field $S$ represents a \textbf{"Wrinkle" or "Tear" in the time flow}. Unlike a point mass that fades away, a topological defect carries a global constraint. The stress in the vacuum ($\nabla S$) cannot relax faster than $1/r$ due to the topology of the Inception World.

\begin{itemize}
    \item \textbf{Gradient Profile:} The "drag" on time decays slowly:
    \begin{equation}
        |\nabla S| = \frac{\eta_S}{r}.
    \end{equation}
    \item \textbf{Comparison:} This is a much slower decay than Newtonian gravity ($1/r^2$). This "long-range drag" is the origin of the mass discrepancy in the outer regions of galaxies.
\end{itemize}

\subsection{Geometric Dark Matter Halo}

We identified the energy density of the dark sector with the gradient energy of the field:
\begin{equation}
    \rho_{\text{DM}} \equiv \frac{\sigma_{\text{vac}}}{2} (\nabla S)^2.
\end{equation}
Substituting the gradient profile $|\nabla S| = \eta_S / r$:
\begin{equation}
    \rho_{\text{DM}}(r) = \frac{\sigma_{\text{vac}}}{2} \left( \frac{\eta_S}{r} \right)^2 = \frac{\sigma_{\text{vac}} \eta_S^2}{2} \frac{1}{r^2}.
\end{equation}

\paragraph{The 1/$r^2$ Profile:}
This result is profound. Without introducing any new particles, the field theory predicts a "Halo" of energy density that falls off as $1/r^2$.
This is mathematically identical to the density profile of a \textbf{Singular Isothermal Sphere}, which is known to produce perfectly flat rotation curves.

\begin{quote}
    \textbf{Conclusion:} The "Dark Matter Halo" is not a cloud of invisible particles. It is the \textbf{elastic stress energy} of the vacuum stored in the topological wrinkle surrounding the galaxy.
\end{quote}

\subsection{Flat Rotation Curves}

The orbital velocity $v(r)$ of a test particle in this effective halo is determined by the enclosed mass $M_{\text{DM}}(r)$.
\begin{equation}
    M_{\text{DM}}(r) = \int_0^r 4\pi x^2 \rho_{\text{DM}}(x) \, dx = \int_0^r 4\pi x^2 \left( \frac{\mathcal{A}}{x^2} \right) dx = 4\pi \mathcal{A} r,
\end{equation}
where $\mathcal{A} = \sigma_{\text{vac}} \eta_S^2 / 2$ is a constant related to the tension of the wrinkle.
The circular velocity is then:
\begin{equation}
    v^2(r) = \frac{G M_{\text{DM}}(r)}{r} = \frac{G (4\pi \mathcal{A} r)}{r} = 4\pi G \mathcal{A} = \text{const}.
\end{equation}

This demonstrates that \textbf{flat rotation curves are a generic prediction} of the Inception World hypothesis. They are not anomalies; they are the expected behavior of gravity in a universe containing topological defects.

\subsection{The Anchor Effect and Baryonic Coupling}

Why do these wrinkles form around galaxies?
According to the \textbf{Anchor Condition} (Chapter 6), baryonic mass acts as the pinning center for these defects.
\begin{itemize}
    \item A galaxy is not just a collection of stars; it is a "knot" in the time-flow.
    \item The baryonic mass $M_b$ determines the core radius $r_c$ and the amplitude of the wrinkle.
    \item This tight coupling explains the \textbf{Renzo's Rule} and the \textbf{Radial Acceleration Relation (RAR)}: the detailed features of the baryonic distribution are imprinted on the rotation curve because the baryons are physically shaping the defect.
\end{itemize}

\subsection{Cosmological Cutoff}

The logarithmic potential ($S \sim \ln r$) and the linear mass growth ($M \sim r$) cannot continue indefinitely, as the total energy would diverge.
In the Enchan cosmology, this divergence is naturally cut off by:
\begin{enumerate}
    \item \textbf{Neighboring Defects:} The wrinkle of one galaxy eventually merges with the wrinkles of its neighbors (cluster scale).
    \item \textbf{Cosmic Horizon:} The finite size of the observable universe (or the Inception bubble) provides a hard cutoff.
\end{enumerate}
This suggests that "Dark Matter" effects are most prominent at galactic to cluster scales, where the $1/r$ gradient can dominate, but must saturate at cosmological scales.