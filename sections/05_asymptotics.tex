%==============================================================================
% Section 5: Asymptotic Structure and Galaxy-Scale Scaling Relations
%==============================================================================

\chap{Asymptotic Structure and Galaxy-Scale Scaling Relations}

This chapter summarizes the asymptotic behavior of the effective field description
in the quasi-static regime and derives scaling relations relevant for galaxy rotation curves.
The discussion is restricted to weak-field, large-radius behavior.

\subsection{Quasi-static limit}

In the quasi-static regime, the effective field equation stated in Chapter~4 admits a
Poisson-like form. For the purpose of asymptotic scaling, we consider the outer-region
limit in which source variations are slow and the solution is dominated by the field's
large-radius behavior:
\begin{equation}
\nabla^2 S \simeq \mathcal{S}(S;\,\text{parameters}),
\end{equation}
where $\mathcal{S}$ denotes the effective source/potential structure.
The precise form used in later chapters is fixed by the phenomenological calibration.

\subsection{Logarithmic asymptotic profile}

A convenient and widely used asymptotic profile for producing flat rotation behavior
is a logarithmic field configuration,
\begin{equation}
S(r) \simeq \eta_S \ln\!\left(\frac{r}{r_c}\right),
\label{eq:S_log}
\end{equation}
where $\eta_S$ is a dimensionless amplitude and $r_c$ is a characteristic scale.
From Eq.~\eqref{eq:S_log} one obtains the radial gradient magnitude
\begin{equation}
\left|\nabla S\right| = \frac{\eta_S}{r}.
\label{eq:grad_1overr}
\end{equation}

Using the working relation $g_{\rm tot} \equiv c^2 \nabla S$,
Eq.~\eqref{eq:grad_1overr} corresponds to an outer-region acceleration scaling
$g_{\rm tot}\propto 1/r$.

\subsection{Effective halo-like density scaling}

For bookkeeping, it is useful to associate an effective density profile with the
field-gradient sector by defining
\begin{equation}
\rho_{\rm eff}(r) \propto \left|\nabla S\right|^2 .
\end{equation}
Substituting Eq.~\eqref{eq:grad_1overr} yields the characteristic scaling
\begin{equation}
\rho_{\rm eff}(r) \propto \frac{1}{r^2}.
\label{eq:rho_1overr2}
\end{equation}
This is the same radial scaling as the singular isothermal sphere profile used as a
standard phenomenological representation of flat rotation curves.

\subsection{Flat rotation scaling}

If the outer-region dynamics are dominated by an effective density profile of the form
$\rho_{\rm eff}\propto r^{-2}$, then the enclosed effective mass scales linearly with radius,
$M_{\rm eff}(r)\propto r$.
The corresponding circular velocity satisfies
\begin{equation}
v^2(r) \propto \frac{M_{\rm eff}(r)}{r} \simeq \text{const},
\end{equation}
which is the standard flat-rotation scaling.

\subsection*{Remark}

This chapter isolates the asymptotic scaling structure used in later derivations.
The connection between $(\eta_S, r_c)$ and baryonic inputs, and the calibration of the
effective response function used for observational benchmarks, are addressed in
Chapters~6--7.
