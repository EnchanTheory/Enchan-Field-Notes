%==============================================================================
% Enchan Field Notes (Theoretical Framework)
% Copyright (c) 2025 Mitsuhiro Kobayashi
%
% This work (textual content, PDF) is licensed under a
% Creative Commons Attribution-NonCommercial 4.0 International License (CC BY-NC 4.0).
% To view a copy of this license, visit http://creativecommons.org/licenses/by-nc/4.0/
%
% The LaTeX source code structure itself is available under the MIT License.
%==============================================================================

\documentclass[11pt,a4paper]{article}

%===========================
% Packages
%===========================
\usepackage[utf8]{inputenc}
\usepackage[T1]{fontenc}
\usepackage{mathptmx}           % Times-like font
\usepackage{geometry}
\geometry{margin=1in}
\usepackage{amsmath,amssymb,amsthm}
\usepackage{bm}
\usepackage{setspace}
\usepackage{hyperref}
\usepackage{booktabs}           % For nicer tables

\hypersetup{
    colorlinks=true,
    linkcolor=blue,
    citecolor=blue,
    urlcolor=blue
}

%===========================
% Helpers
%===========================
% chapter-like section with page break
\newcommand{\chap}[1]{\clearpage\section{#1}}

%===========================
% Title / Author
%===========================
\title{\textbf{Enchan Field Notes v0.2.3}\\[0.5em]
\large Emergent Spacetime, Geometric Dark Matter, and Rotational Control}

\author{Mitsuhiro Kobayashi\\[0.25em]
Tokyo, Japan\\
\texttt{enchan.theory@gmail.com}}

\date{\today}

%===========================
% Document
%===========================
\begin{document}

\maketitle
\thispagestyle{empty}

\begin{abstract}
\noindent
These notes collect a minimal, field-theoretic realization of the Enchan framework.
The central idea is that spacetime geometry is not fundamental but emerges as a
stabilized long-lived configuration of a primordial fluctuation field. In this picture, dark
matter is not a new particle species but the geometric imprint of persistent
defects---``spacetime wrinkles''---in an emergent spatial order parameter $S$.

Previous updates (v0.2.2) incorporated the role of rotation in stability, drawing on Kerr black hole mechanics.
\textbf{In this v0.2.3 update}, we append a note on recent experimental constraints. Specifically, we summarize the null results from a search for macroscopic rotational coupling using ground-based interferometric limits, which, within the tested band and sensitivity limits, rule out the ``human-scale strong-coupling'' variant of the hypothesis.

The Enchan framework is \emph{not} intended to deny the Standard Model or general
relativity. It should be read as a speculative ``source-code'' layer proposed
beneath or alongside existing effective theories, especially in how we interpret
the dark-matter sector.

Starting from a conceptual set of effective degrees of freedom---fluctuations $F$, emergent space $S$,
time $T$, localized excitations $P$, and a geometric dark sector $D$---we derive
a simple Lagrangian for the Enchan field,
\[
\mathcal{L} = \tfrac{1}{2}(\partial_\mu S)(\partial^\mu S) + F S - V(S),
\]
and obtain the equation of motion
\[
\square S = F - \frac{\partial V}{\partial S}.
\]
In the static galactic regime, we introduce a hypothesis that asymptotic
configurations with $\|\nabla S\| \propto 1/r$ generate an effective density
profile $\rho(r) \propto 1/r^2$, which yields flat rotation curves
$v(r) = \text{const}$ without particulate dark matter. We summarize the
conceptual structure, its minimal field realization, and possible observational
signatures, and we make explicit which elements are assumptions rather than
derived results. This v0.2.3 version is intended as a technical backbone and
ledger for future refinements.
\end{abstract}

\clearpage
\tableofcontents
\clearpage

%===========================
% Chapter 1
%===========================
\chap{Overview and Roadmap}

The Enchan framework starts from a simple but radical premise: spacetime geometry
is not fundamental. Instead, the spatial structure we experience is a stabilized
configuration of underlying microscopic fluctuations.
In this view, gravity and dark matter are not added on top of a pre-given metric;
rather, they emerge from the way this spatial order forms, relaxes, and fails
to fully relax.

A companion essay (\emph{``Dark Matter as Spacetime Wrinkles''}, submitted to
the Gravity Research Foundation 2026) formulates this idea in conceptual and
astrophysical terms: dark matter is identified with persistent topological
defects (\emph{wrinkles}) in an emergent metric, whose $1/r^2$ energy-density
profile explains flat galactic rotation curves and whose solitonic dynamics
are potentially consistent with cluster collisions such as the Bullet Cluster.

The purpose of these notes is different but complementary:
\begin{itemize}
\item to spell out the conceptual backbone of the framework, relating
fluctuations, space, time, matter, and dark matter;
\item to present a minimal scalar-field model---the \emph{Enchan field} $S$---
that realizes these concepts at an effective level;
\item to show explicitly how geometric defects in $S$ can reproduce the
phenomenology usually attributed to particle dark matter;
\item to delineate which parts of the construction are assumptions,
which are derived, and which are left as open problems;
\item to record, in a programmatic way, how these notes relate to a separate
conceptual essay (GRF submission) and to a device-level patent
application based on the same underlying picture.
\end{itemize}

\subsection*{Scope and relation to existing physics}

Throughout these notes, ``Enchan theory'' is intended as a speculative layer
\emph{beneath or alongside} existing frameworks:

\begin{itemize}
\item The Standard Model and general relativity are treated as effective
theories that remain valid wherever they have been observationally
tested.
\item The Enchan framework proposes a different microscopic picture of how
spacetime and the dark sector might emerge, but it is \emph{not}
designed to contradict well-established empirical results.
\item In particular, ordinary particles such as electrons and protons are
assumed to exist as usual; they are accommodated in the Enchan scheme
as part of the localized excitation sector $P$.
\item \textbf{Technological implementations are explicitly out of scope.}
While the theoretical concepts (e.g., rotational control of stability) may
inspire engineering applications, this document is strictly limited to
cosmological and field-theoretic descriptions. Specific device architectures,
sensing algorithms, or industrial applications are distinct from this
framework and are reserved for separate documentation.
\end{itemize}

The structure of these notes is as follows.
Chapter~2 summarizes the conceptual structure of Enchan theory.
Chapter~3 discusses the physical motivation for the role of rotation,
based on Kerr black hole mechanics.
Chapter~4 introduces the field-theoretic model for $S$ and derives its equation
of motion. Chapter~5 discusses asymptotic configurations and their implication
for galactic rotation curves.
Chapter~6 outlines observational and theoretical directions.
Chapter~7 concludes with a summary.
Chapter~8 provides a programmatic mapping between these notes, the essay, and the patent.

%===========================
% Chapter 2 (PUBLIC-FACING)
%===========================
\chap{Conceptual Structure of the Enchan Framework}

In this chapter we summarize the conceptual backbone of the Enchan framework
in a way that is compatible with standard field-theoretic language.
We deliberately avoid committing to any specific microscopic ``axioms'' or
fixed-point equations; those belong to a more speculative, internal layer
of the programme. Here we only introduce the effective degrees of freedom
that will be used in the rest of these notes.

\subsection*{Core ingredients}

At the level of this document, the Enchan framework is built from the
following ingredients:

\begin{itemize}
\item A \emph{fluctuation sector} $F(x)$, representing pre-geometric or
microscopic inhomogeneities that act as a coarse-grained source for
spatial structure.

\item A \emph{rotational degree of freedom} $\Omega(x)$, capturing the
irreducibly anisotropic, vortical component of the fluctuations.
This is motivated by the role of spin and frame dragging in Kerr
spacetimes: rotation is treated as a constitutive part of how
stability domains are formed.

\item An \emph{emergent spatial field} $S(x)$, which summarizes the
large-scale ordering of space. In the minimal model of
Chapter~4, $S$ is realized as a scalar field whose dynamics
are driven by $F$ and stabilized by a self-interaction potential
$V(S)$.

\item A \emph{geometric dark sector} $D(x)$, defined at the effective
level as a functional of $S$, and in the simplest realization
identified with the norm of its spatial gradient:
\begin{equation}
D(x) \equiv \|\nabla S(x)\|.
\end{equation}
This object is interpreted as a measure of ``spacetime wrinkles''
or scars in the emergent spatial order, and plays the role
usually attributed to dark matter.

\item A \emph{localized excitation sector} $P(x)$, which denotes
ordinary matter and other localized structures (e.g.\ stars,
galaxies, compact objects). In the broader Enchan picture,
these are understood as localized defects or concentrations
that live on top of the background geometry encoded by $S$
and $D$, but their detailed realization is left open in
the present notes.

\item An \emph{effective time direction} $T$, associated not with a
fundamental parameter but with gradients and flows of $S$ at
large scales. We will not attempt to construct $T$ explicitly
in this version; it suffices to note that, in the Enchan
perspective, macroscopic time is tied to the evolution of
spatial order rather than imposed \emph{a priori}.
\end{itemize}

In short, $F$ and $\Omega$ provide the driving environment, $S$ captures
emergent spatial order, $D$ measures its residual geometric defects,
$P$ stands for localized excitations, and $T$ encodes the effective
time direction associated with the evolution of $S$.

\subsection*{From concepts to an effective field model}

The more speculative, microscopic side of the Enchan proposal includes
various fixed-point relations between these quantities. In this document
we do \emph{not} assume or specify those relations in detail. Instead,
we adopt the following effective stance:

\begin{itemize}
\item The fluctuation sector $F$ is treated as an external source.
Its precise microscopic origin is left unspecified, but it is
assumed to possess well-defined large-scale statistics.

\item The rotational degree $\Omega$ is regarded as a control parameter
that can modify the stability properties of emergent configurations
of $S$, in analogy with how the spin parameter $a_*$ modifies the
ISCO radius in Kerr spacetimes (Chapter~3).

\item The field $S$ is modeled as a scalar order parameter whose
equation of motion follows from a standard Lagrangian of the
form introduced in Chapter~4:
\[
\mathcal{L} = \tfrac{1}{2}(\partial_\mu S)(\partial^\mu S) + F S - V(S).
\]

\item The geometric dark sector $D$ is identified with a functional
of $S$; in the simplest realization used here we take
$D = \|\nabla S\|$ and associate the dominant dark-matter energy
density with the gradient energy of $S$:
\[
\rho_D \propto (\nabla S)^2.
\]
\end{itemize}

This is deliberately modest: we do not claim to know the full microscopic
dynamics of $(F,\Omega,S)$, nor do we derive time from first principles.
Instead, we specify just enough structure to construct a concrete field
model (Chapter~4) and to explore its asymptotic implications for galactic
dynamics (Chapter~5).

\subsection*{Internal versus public layers (programmatic note)}

The broader Enchan programme entertains a more detailed set of symbolic
relations between fluctuations, space, dark sectors and localized
excitations, including fixed-point expressions that are intended as an
internal ``source-code'' layer of the framework.

Those internal relations are \emph{not} required for the field-theoretic
analysis presented here and are therefore deliberately omitted from this
public-facing version of the notes. The present chapter should be read as
a conceptual distillation that is sufficient to connect the Enchan picture
to standard effective field theory and to observationally accessible
questions in cosmology.

%===========================
% Chapter 3 (NEW in v0.2.x)
%===========================
\chap{Physical Motivation: Rotation and Stability}

The Enchan conceptual structure posits that emergent spatial
order depends critically on a rotational component $\Omega$. Before proceeding to
the effective scalar field model, we briefly review the standard general relativistic
justification for this dependency.

In general relativity, the ``shape'' and ``depth'' of a gravity well are determined
not only by mass $M$ but also by angular momentum $J$. This is described by the
Kerr metric, which provides a precise analog for how rotation modifies the stability
domain of the Enchan field.

\subsection{The Spin Parameter and Geometry}

In the Kerr spacetime geometry, the influence of rotation is quantified by the
dimensionless spin parameter $a_*$:
\begin{equation}
a_* \equiv \frac{cJ}{GM^2}, \qquad -1 \le a_* \le 1.
\end{equation}
When $a_* \neq 0$, the spacetime metric ceases to be static and acquires a
cross-term $dt\,d\phi$, leading to \emph{frame dragging} (the Lense--Thirring effect).
In the slow-rotation, weak-field limit, the dragging angular velocity is
\begin{equation}
\Omega_{\rm LT}(r) \simeq \frac{2GJ}{c^2 r^3}.
\end{equation}
This indicates that the ``vacuum'' itself is twisted by the rotating source, creating
a preferred orientation for stability.

\subsection{Modification of Stability Limits (ISCO)}

The most significant feature for Enchan theory is how rotation alters the
\emph{Innermost Stable Circular Orbit} (ISCO). The ISCO radius $r_{\rm ISCO}$ defines
the boundary beyond which stable orbits cannot exist (i.e., matter plunges into the
horizon).

For a non-rotating Schwarzschild black hole ($a_* = 0$), this limit is:
\begin{equation}
r_{\rm ISCO} = 6 \left( \frac{GM}{c^2} \right).
\end{equation}
However, for a maximally rotating Kerr black hole ($a_* \to 1$) in the prograde direction,
this limit shrinks drastically:
\begin{equation}
r_{\rm ISCO} \to 1 \left( \frac{GM}{c^2} \right).
\end{equation}
The exact dependency is given by the standard Kerr relations:
\begin{align}
r_{\rm ISCO}^{\pm} &= \frac{GM}{c^2} \left[ 3 + Z_2 \mp \sqrt{(3-Z_1)(3+Z_1+2Z_2)} \right], \\
Z_1 &= 1 + (1-a_*^2)^{1/3} [(1+a_*)^{1/3} + (1-a_*)^{1/3}], \nonumber \\
Z_2 &= \sqrt{3a_*^2 + Z_1^2}. \nonumber
\end{align}

\subsection{Interpretation in the Enchan Framework}

This relativistic result serves as the core physical motivation for our model:
\begin{quote}
\emph{Increasing the rotational control (spin) allows stable structures to exist
much deeper within a potential well than would be possible in a static field.}
\end{quote}
In the Enchan language, the parameter $a_*$ corresponds to the intensity of the
$\Omega$ control field. By modulating $\Omega$, the system can actively modify the
effective stability radius of the emergent space $S$, effectively ``pushing back''
the breakdown of geometry (the horizon). This justifies treating $\Omega$ as a fundamental
control parameter alongside the fluctuation amplitude $F$.

\subsection{Mapping GR Concepts to Enchan Variables}

Based on the correspondence above, we establish the following conceptual mapping between
standard General Relativity (GR) and the Enchan framework (Table \ref{tab:mapping}).

\begin{table}[h]
\centering
\caption{Correspondence between GR parameters and Enchan control variables.}
\label{tab:mapping}
\renewcommand{\arraystretch}{1.2}
\begin{tabular}{@{}lp{6cm}l@{}}
\toprule
\textbf{GR Symbol} & \textbf{Physical Meaning (GR)} & \textbf{Enchan Analog} \\ \midrule
$M$ & Mass of the central body & Basic scale of the potential well \\
$a_*$ (or $J$) & Spin / Angular Momentum & Intensity of $\Omega$ (Rotational Control) \\
$r_{\rm ISCO}$ & Innermost Stable Circular Orbit & Stability limit of the emergent order $S$ \\
$\Omega_{\rm LT}$ & Frame-dragging frequency & Intrinsic resonant frequency of the field \\
\bottomrule
\end{tabular}
\end{table}

%===========================
% Chapter 4
%===========================
\chap{Minimal Enchan Field Model}

In the Enchan framework, the emergent spatial order $S$ arises as a stabilized
long-lived configuration of a primordial fluctuation field $F$. A natural way to capture
this at a coarse-grained level is to model $S(x)$ as a scalar order parameter
whose dynamics is driven by $F$ and stabilized by a self-interaction potential.

\subsection{Universality and independence from microscopic details}

Following the logic of modern renormalization-group theory, we assume:
\begin{quote}
\emph{The large-scale dynamics of $S$ depends only on the universality class
of $F$, not on its detailed microscopic realization.}
\end{quote}
This principle ensures that the emergent theory is robust and does not rely
on fine-tuning of the underlying fluctuation model. In these notes we treat
$F$ as an effective source term with specified large-scale statistics, without
committing to a particular microscopic origin.

\subsection{Lagrangian formulation}

We introduce the \emph{Enchan field} $S(x)$ via the following Lagrangian
density:
\begin{equation}
\boxed{
\mathcal{L} \;=\;
\frac{1}{2}(\partial_\mu S)(\partial^\mu S)
\;+\; F S
\;-\; V(S)
}
\label{eq:Lagrangian}
\end{equation}
where $F$ plays the role of a source term coupling to $S$, and $V(S)$ is a
self-interaction potential.

We use the standard convention
\begin{equation}
\mathcal{L} = T - V,
\end{equation}
so that the Hamiltonian is bounded below and the potential exerts a restoring
force. The term $F S$ represents the coupling between the primordial
fluctuation source and the emergent spatial field. The potential $V(S)$
controls stability, vacuum selection, and defect formation.

\subsection{On the form of the potential \texorpdfstring{$V(S)$}{V(S)}}

Just as the Higgs field's ``Mexican-hat'' potential is fixed by the requirement
of spontaneous symmetry breaking, the specific form of $V(S)$ in the Enchan
theory will be constrained by the requirement that $S$ supports stable
topological defects (``spacetime wrinkles''):
\begin{equation}
\text{stable geometric wrinkles}
\;\Longleftrightarrow\;
\text{topological defects of } S.
\end{equation}
These defects are intended to realize the dark-matter sector defined by
$D \equiv \|\nabla S\|$.

In this minimal model we do not specify a unique form of $V(S)$.
Instead, we assume that $V(S)$ belongs to a class of potentials that admit
nontrivial topological configurations and whose far-field behavior can
support asymptotic configurations with $\|\nabla S\| \propto 1/r$. Constructing
explicit examples of such potentials in full three-dimensional geometry is an
important open problem.

\subsection{Euler--Lagrange equation of motion}

Varying Eq.~\eqref{eq:Lagrangian} with respect to $S$ gives
\begin{align}
\frac{\partial \mathcal{L}}{\partial (\partial_\mu S)}
&= \partial^\mu S, \\
\frac{\partial \mathcal{L}}{\partial S}
&= F - \frac{\partial V}{\partial S}.
\end{align}
Thus the Euler--Lagrange equation is
\begin{equation}
\partial_\mu \partial^\mu S
\;=\;
F - \frac{\partial V}{\partial S}.
\end{equation}
Equivalently,
\begin{equation}
\boxed{
\square S \;=\; F \;-\; \frac{\partial V}{\partial S}
}
\label{eq:FieldEq}
\end{equation}
where $\square = \partial_\mu \partial^\mu$ is the d'Alembert operator.

This expresses the Enchan principle in dynamical form:
\begin{equation}
\text{space is driven by fluctuations and stabilized by potential curvature.}
\end{equation}

For the purposes of these notes, we treat the energy density associated with
$S$ in the usual way, with a dominant contribution from the gradient term
in regimes where the field varies slowly in time:
\begin{equation}
\rho_S \;\sim\; \frac{1}{2} (\nabla S)^2 \;+\; V(S).
\end{equation}
In the next chapter we focus on regimes where $(\nabla S)^2$ dominates and
explore the consequences for galactic dynamics.

\subsection*{Status of dimensional analysis in v0.2.3}

In this v0.2.3 version we do not yet fix the mass dimensions of $F$, $S$ and
$V(S)$ in a fully systematic way. The coupling term $F S$ in
Eq.~\eqref{eq:Lagrangian} should ultimately be accompanied by an explicit
coupling constant so that the total mass dimension of $\mathcal{L}$ is
$\mathrm{mass}^4$ in four spacetime dimensions, as in standard relativistic
field theory.

Here we treat the model as an effective description and leave the precise
normalization and dimensional counting to a future, more complete version
(v0.3 and beyond), where $[S]$, $[F]$ and the mass-dimension of $V(S)$ will be
fixed and checked systematically. The present notes should therefore be read
as specifying the \emph{structure} of the coupling, not its fully normalized
form.

\subsection*{Symmetries and conservation laws (programmatic remark)}

At the level of the minimal field model, we treat $S$ as a Lorentz scalar and
assume that the Lagrangian~\eqref{eq:Lagrangian} is Poincar\'e invariant in the
usual sense. This guarantees the standard Noether currents associated with
translations and Lorentz transformations, and thus energy--momentum conservation
for the Enchan field in regimes where the background is approximately
Minkowskian.

At the same time, the broader Enchan framework allows for the possibility that
the pre-geometric fluctuation sector $(\epsilon,F)$ lives at a more microscopic
level where different symmetries---or their breaking---may be relevant. In this
v0.2.3 version we do not attempt to specify these microscopic symmetries. The
question of how the symmetries of $(\epsilon,F)$ flow to the effective
Poincar\'e symmetry of $S$ is deferred to future work and is regarded as a key
part of a more complete Enchan cosmology.

%===========================
% Chapter 5
%===========================
\chap{Asymptotic Structure and Galactic Dynamics}

On galactic scales, the background configuration of $S$ is expected to evolve
slowly compared to orbital timescales. As a first approximation, we therefore
neglect time derivatives and take a static limit of Eq.~\eqref{eq:FieldEq}:
\begin{equation}
\square S \approx -\nabla^2 S,
\end{equation}
so that
\begin{equation}
\nabla^2 S
\;=\;
\frac{\partial V}{\partial S} - F.
\label{eq:PoissonLike}
\end{equation}

We are interested in asymptotic regimes where the influence of $F$ and
$V'(S)$ on the right-hand side combines to support configurations with a
specific large-radius behavior of $\nabla S$.

\subsection{Hypothesis on galactic asymptotics}

In this minimal version of the Enchan field model, we do \emph{not} derive the
far-field behavior of $S$ from a fully specified potential $V(S)$ and source
$F$. Instead, we introduce the following working hypothesis:

\medskip\noindent
\textbf{Hypothesis 1 (galactic asymptotics).}
\emph{For a suitable class of potentials $V(S)$ and large-scale effective
sources $F$, there exist stable or long-lived configurations of $S$ in three
spatial dimensions such that, at large radii,}
\begin{equation}
\|\nabla S\|(r) \;\propto\; \frac{1}{r}
\qquad (r \to \infty).
\label{eq:gradAsymptotic}
\end{equation}

\medskip
This hypothesis is motivated by known examples of topological defects,
such as global monopoles and related configurations, whose energy densities
scale as $1/r^2$ at large radii. However, in the present notes we treat
Eq.~\eqref{eq:gradAsymptotic} explicitly as an \emph{assumption}, not as a
proved theorem.

\subsection{Geometric dark sector and effective density profile}

By definition,
\begin{equation}
D(r) = \|\nabla S\|,
\end{equation}
and we identify the dominant dark-matter contribution to the energy density
with the gradient energy of $S$,
\begin{equation}
\rho_D(r) \;\propto\; (\nabla S)^2.
\end{equation}
If Hypothesis~1 holds, then at large radii we have
\begin{equation}
\rho_D(r) \;\propto\; \frac{1}{r^2}.
\end{equation}
In a Newtonian approximation, the enclosed mass $M(r)$ then grows linearly
with $r$:
\begin{equation}
M(r)
= \int_0^r 4\pi x^2 \rho_D(x)\,dx
\;\propto\;
\int_0^r 4\pi x^2 \frac{1}{x^2}\,dx
\;\propto\; r,
\end{equation}
so that the circular velocity of a test particle is
\begin{equation}
v^2(r) = \frac{G M(r)}{r} = \text{const}.
\end{equation}
Thus, flat rotation curves emerge naturally from the asymptotic geometry of
the Enchan field $S$, without the need for particulate dark matter, provided
that the defect spectrum of $S$ includes configurations with
$\|\nabla S\| \propto 1/r$ as in Hypothesis~1.

\subsection{Status and limitations of the asymptotic assumption}

It is important to emphasize that, at this stage, the condition
$\|\nabla S\| \propto 1/r$ is an \emph{asymptotic requirement} rather than a
fully derived result. In particular:
\begin{itemize}
\item In three spatial dimensions, the simplest harmonic solutions of
$\nabla^2 S = 0$ yield $S \sim \text{const} + B/r$, not $\log r$.
Achieving $\|\nabla S\| \propto 1/r$ in a fully consistent
defect configuration generally requires a nontrivial interplay
between the potential $V(S)$, the source term $F$, and the geometry.
\item Known examples such as global monopoles and related topological
defects provide useful guidance but do not yet uniquely fix the
Enchan potential or defect spectrum.
\end{itemize}
A central task for future work is therefore to construct explicit, fully
three-dimensional defect solutions of Eq.~\eqref{eq:PoissonLike} that realize
the desired asymptotics and satisfy cosmological and astrophysical
constraints.

%===========================
% Chapter 6
%===========================
\chap{Observational Signatures and Open Problems}

The minimal Enchan field model sketched here is deliberately modest in scope,
but it already suggests several concrete directions for further theoretical
and observational study.

\subsection{Rotation-curve structure}

Under Hypothesis~1, the asymptotic condition $\|\nabla S\| \propto 1/r$
implies a robust $1/r^2$ density profile at large radii. Deviations from this
scaling near a core scale $r_c$, where the influence of $V(S)$ and $F$ is
stronger, may lead to characteristic features in high-resolution galactic
rotation curves. Comparing such features with existing data offers a near-term
test of the model.

\subsection{Cluster dynamics and lensing}

If the dark sector $D = \|\nabla S\|$ dominates the gravitational potential
on cluster scales, the spatial distribution of $S$ should be imprinted in
weak- and strong-lensing maps. Systems such as the Bullet Cluster, which
are often used to argue for particulate dark matter, become important
testbeds: solitonic or defect-like configurations of $S$ should pass through
baryonic plasma without drag, potentially reproducing the observed separation
between gas and lensing mass.

%==============================================================================
% Added for v0.2.3 update
%==============================================================================
\subsection{Constraints from macroscopic rotational-coupling searches (v0.2.3 note)}

Recent feasibility studies and null searches using public interferometer data have placed important constraints on the scaling of the geometric coupling constants postulated in the Enchan framework.

\subsubsection*{Hypothesis tested (restricted form)}
We investigated a restricted ``strong-coupling'' variant of the Enchan programme in which macroscopic rotation or vibration of ordinary matter induces an additional, non-GR geometric response that is:
\begin{enumerate}
    \item Detectable as a non-reciprocal optical phase bias in interferometric readouts (Sagnac-like effect), and/or
    \item Manifests as a narrowband, approximately stationary feature in strain-sensitive instruments within a representative human-scale frequency band ($10$--$100$\,Hz).
\end{enumerate}

\subsubsection*{Methodology}
Two complementary constraint routes were pursued to bound the coupling strength:

\paragraph{Detectability studies for interferometric phase readouts.}
Using standard interferometric scaling relations, we quantified how thermally induced non-reciprocity (Shupe effect) and polarization-related visibility fluctuations bound achievable sensitivity in practical ground-based configurations. The goal of this study was not device design, but to estimate whether the hypothesized coupling could plausibly rise above dominant, well-known noise terms in realistic environments.

\paragraph{Search for narrowband stationary features in public strain data.}
We analyzed representative segments of publicly available high-sensitivity interferometer data (public LIGO/Virgo strain data via GWOSC \cite{GWOSC}) and searched the $10$--$100$\,Hz band for persistent narrow spectral features consistent with the restricted hypothesis. Candidate lines were cross-checked against known instrumental artifacts (e.g., scattering arches, calibration lines, and suspension resonances) to assess whether any statistically compelling anomaly remained.

\subsubsection*{Results (null within scope)}
Across the tested parameter space and analysis settings, no evidence was found for an additional strong geometric response beyond known instrumental/systematic features. 
In particular, the detectability study indicates that dominant thermal non-reciprocity places stringent practical bounds on how large any putative effect could be before it becomes indistinguishable from environmental drift in compact, ground-based configurations. Similarly, strain data analysis yielded no candidates exceeding typical strain noise levels in that band (order $10^{-23}$--$10^{-22}/\sqrt{\mathrm{Hz}}$) that could not be vetted as instrumental noise.

\subsubsection*{Conclusion and implication for model-building}
The above results do not falsify the broader Enchan hypothesis space. They do, however, rule out---within the tested band, sensitivity targets, and analysis assumptions---the specific ``human-scale strong-coupling'' realization. Future versions of the theory should therefore:
\begin{itemize}
    \item Treat the coupling strength as bounded from above by these null results (i.e., the effect must be weaker than the current observational limits).
    \item Shift emphasis toward either microphysical/quantum regimes or genuinely strong-gravity/cosmological regimes where the relevant scales may differ by many orders of magnitude.
\end{itemize}

\paragraph{Remark on ``resonance'' in astrophysical strong gravity.}
Damped post-merger oscillations (``ringdown'') observed in compact-binary signals are consistent with the quasi-normal-mode response of perturbed black-hole spacetimes in general relativity. This phenomenon reflects a strong-gravity transient excited by a large perturbation and should not be conflated with a laboratory-scale, externally driven macroscopic rotational coupling. Consequently, similarity in frequency band alone (e.g., $10$--$100$\,Hz) does not imply scale invariance; the relevant control parameters are the strength of curvature and relativistic compactness, which differ by many orders of magnitude between astrophysical mergers and terrestrial rotating apparatus.

\subsection{Cosmological evolution}

Embedding Eq.~\eqref{eq:FieldEq} in a cosmological FRW background would allow
a systematic study of how the Enchan field contributes to structure formation.
Key questions include:
\begin{itemize}
\item Can an ensemble of Enchan defects seed early structures in a way that
alleviates current tensions in early-galaxy observations?
\item How does the defect network evolve with cosmic time, and does it
respect constraints from the cosmic microwave background and large-scale
structure?
\end{itemize}

\subsection*{Limitations of the present v0.2.3 model}

The discussion in this chapter is intentionally qualitative. In particular:

\begin{itemize}
\item We do not yet embed Eq.~\eqref{eq:FieldEq} into the full Einstein
equations. A consistent treatment of lensing, light-cone structure,
and cosmological evolution requires specifying how $T_{\mu\nu}[S]$
couples to the metric $g_{\mu\nu}$, or whether $g_{\mu\nu}$ is itself
an emergent functional of $S$. This is left open in v0.2.3.
\item We do not derive detailed constraints from the cosmic microwave
background or large-scale structure. Known bounds on defect networks
(e.g.\ cosmic strings, global monopoles) provide important guidance,
but a quantitative comparison for Enchan-type defects remains to be
carried out.
\item Beyond the generic prediction of asymptotically flat rotation curves
from a $1/r^2$-type gradient energy profile, we do not propose in
v0.2.3 a set of sharp, unique observational signatures that would
distinguish the Enchan framework from other dark-matter models.
\end{itemize}

These limitations are not accidental but reflect the intended scope of this
version: to provide a minimal, technically standard field-theoretic backbone
on top of which more detailed cosmological and observational analyses can
be built.

%===========================
% Chapter 7
%===========================
\chap{Conclusion}

We have presented a minimal, field-theoretic realization of the Enchan
framework, in which an emergent scalar field $S$ models stabilized spatial
order and its defects play the role of geometric dark matter. Starting from
a compact set of effective degrees of freedom relating fluctuations, space, time, localized
excitations, and a dark sector, we introduced a simple Lagrangian
\[
\mathcal{L} = \tfrac{1}{2}(\partial_\mu S)(\partial^\mu S) + F S - V(S),
\]
derived the equation of motion
\[
\square S = F - \frac{\partial V}{\partial S},
\]
and, under a stated hypothesis on the asymptotic behavior of $\nabla S$,
showed how configurations with $\|\nabla S\| \propto 1/r$ naturally lead to
a $1/r^2$ density profile and flat galactic rotation curves.

Many ingredients are still missing from a fully developed Enchan cosmology:
a detailed specification of $V(S)$, explicit defect solutions in expanding
backgrounds, a consistent coupling to ordinary matter and to the metric
in general relativity, and a quantitative confrontation with cosmological
data. Nevertheless, the notes demonstrate that the Enchan concepts admit a
mathematically standard and observationally relevant field-theoretic
realization, provided one is willing to treat the key asymptotic behavior
as a hypothesis to be tested rather than a theorem already proved.

From a broader perspective, these notes are intended as a transportable
description and ledger of the framework: they make explicit which parts are
speculative, which parts are conventional field theory, and how the Enchan
picture relates to existing physics. This should allow future readers---human
or machine---to pick up the thread, refine the assumptions, and test the
consequences against data.

%===========================
% Chapter 8
%===========================
\chap{Programmatic Context and Long-Term Roadmap}

This final chapter briefly records how the present notes relate to other
Enchan-related documents, and how they are intended to evolve over time.

\subsection{Multi-format realization of a single hypothesis}

Enchan theory is meant to be a single underlying hypothesis about a possible
``source code'' for the physical universe, realized simultaneously in several
formats:

\begin{itemize}
\item A \emph{conceptual essay} (e.g.\ a Gravity Research Foundation
submission) which presents the emergent-spacetime and geometric
dark-matter picture in narrative and astrophysical terms, with a
focus on existing observations such as galactic rotation curves,
gravitational lensing, and early-galaxy data.
\item The present \emph{Enchan Field Notes}, which express the same picture
in the language of effective field theory, with explicit field-theoretic constructions,
equations of motion, and clearly stated hypotheses.
\item A \emph{device-level patent application}, which asks: assuming the
Enchan picture is approximately correct, what kinds of spacetime-%
modulation or ``metric control'' devices might be conceivable in
principle, even if they are far beyond current engineering?
\item A broader \emph{narrative and observational-literature program},
in which ideas about fluctuations, emergent space, time, and
defects are explored at the level of human perception and story,
as a kind of cognitive laboratory for the same concepts.
\end{itemize}

These are not independent projects but different projections of the same
underlying attempt to model spacetime and dark matter as emergent structures.

\subsection{Role of these notes as a ledger}

Within this multi-format picture, the present notes play a specific role:

\begin{itemize}
\item They serve as a \emph{ledger} or \emph{versioned notebook} in which the
mathematical and physical content of the Enchan framework is recorded
in a compact, technical form.
\item They explicitly mark which elements are assumptions, which are standard
constructions in field theory, and which are hypotheses awaiting
confirmation or refutation.
\item They provide a natural location for incorporating new references,
numerical experiments, and consistency checks over time (e.g.\ v0.2,
v1.0, and beyond).
\end{itemize}

In particular, if future simulations or observations falsify key Enchan
assumptions (for example, if no defect spectrum can reproduce the required
asymptotics within acceptable cosmological bounds), these notes are intended
to record that fact and to provide a starting point for formulating alternative
hypotheses.

\subsection{Long-term roadmap (10--20 year horizon)}

The intended time scale for developing or discarding the Enchan framework as a
physical hypothesis is not one or two years but on the order of one or two
decades. Very schematically, one can imagine the following stages:

\paragraph{Near term (v0.x series).}
\begin{itemize}
\item Clarify dimensional analysis and normalization of the $F S$ coupling,
including explicit mass dimensions for $F$, $S$ and $V(S)$.
\item Construct at least one explicit example of a potential $V(S)$ and
source configuration $F$ that supports nontrivial defects in three
dimensions, and explore them numerically in simplified settings.
\item Refine the axiomatic structure as needed in light of these examples.
\end{itemize}

\paragraph{Medium term (v1.x series).}
\begin{itemize}
\item Embed the Enchan field in an explicit coupling to the metric, via
either $T_{\mu\nu}[S]$ in Einstein's equations or a more radical
emergent-metric construction.
\item Compare the resulting model with galactic rotation curves and lensing
data, at least in simplified or idealized systems.
\item Begin to confront the defect spectrum with basic cosmological
constraints, including the cosmic microwave background.
\end{itemize}

\paragraph{Longer term.}
\begin{itemize}
\item Perform more systematic cosmological simulations of defect networks
and structure formation in Enchan-like models, using them to test the
framework against a wider range of observations.
\item Either progressively tighten the Enchan picture into a more predictive
theory, or, if it fails, document how and where it breaks down and
treat it as a stepping stone toward a more accurate description.
\end{itemize}

In all of these stages, the guiding principle is that Enchan theory is to be
treated as a serious but speculative hypothesis: it should be sharpened,
criticized, and, if necessary, falsified in the usual scientific manner.

\section*{Acknowledgements}

The author thanks informal collaborators and reviewers for their critical
feedback on earlier versions of this framework.

\clearpage
\bibliographystyle{plain}
\begin{thebibliography}{9}

\bibitem{Kibble1976}
T.~W.~B. Kibble,
\newblock Topology of cosmic domains and strings.
\newblock {\em Journal of Physics A: Mathematical and General}, 9(8):1387, 1976.

\bibitem{BarriolaVilenkin1989}
M.~Barriola and A.~Vilenkin,
\newblock Gravitational field of a global monopole.
\newblock {\em Physical Review Letters}, 63(4):341, 1989.

\bibitem{Perivolaropoulos2022}
L.~Perivolaropoulos,
\newblock Frustrated topological defects.
\newblock {\em Physical Review D}, 105(12):124002, 2022.

\bibitem{GWOSC}
R. Abbott et al. (LIGO Scientific Collaboration and Virgo Collaboration),
\newblock Open Data from the First and Second Observing Runs of Advanced LIGO and Advanced Virgo.
\newblock {\em SoftwareX}, 13:100658, 2021.
\newblock \url{https://www.gw-openscience.org}

\end{thebibliography}

\end{document}
