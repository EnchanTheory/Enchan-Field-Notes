%==============================================================================
% Enchan SPARC Rotation-Curve Prediction Test Report
% Copyright (c) 2025 Mitsuhiro Kobayashi
%
% This work (textual content, PDF) is licensed under a
% Creative Commons Attribution-NonCommercial 4.0 International License (CC BY-NC 4.0).
% To view a copy of this license, visit http://creativecommons.org/licenses/by-nc/4.0/
%
% The LaTeX source code structure itself is available under the MIT License.
%==============================================================================

\documentclass[11pt,a4paper]{article}

%===========================
% Packages (match Enchan Field Notes style)
%===========================
\usepackage[utf8]{inputenc}
\usepackage[T1]{fontenc}
\usepackage{mathptmx}
\usepackage{geometry}
\geometry{margin=1in}
\usepackage{amsmath,amssymb}
\usepackage{bm}
\usepackage{setspace}
\usepackage{hyperref}
\usepackage{booktabs}
\usepackage{graphicx}

\hypersetup{
    colorlinks=true,
    linkcolor=blue,
    citecolor=blue,
    urlcolor=blue
}

%===========================
% Helpers
%===========================
\newcommand{\chap}[1]{\clearpage\section{#1}}

%===========================
% Title / Author
%===========================
\title{\textbf{Enchan SPARC Rotation-Curve Prediction Report v0.1}\\[0.5em]
\large Fixed-parameter prediction of $V_{\rm obs}(r)$ from public SPARC mass models}
\author{Mitsuhiro Kobayashi\\[0.25em]
Tokyo, Japan\\
\texttt{enchan.theory@gmail.com}}
\date{\today}

\begin{document}

\maketitle
\thispagestyle{empty}

\begin{abstract}
\noindent
This report documents a minimal, reproducible ``rotation-curve prediction'' test using
public SPARC Newtonian mass-model files (\texttt{Rotmod\_LTG}).
For each radial point we compute the baryonic acceleration proxy
$g_{\rm bar}(r)=\left(V_{\rm gas}^2+\Upsilon_{\rm disk}V_{\rm disk}^2+\Upsilon_{\rm bul}V_{\rm bul}^2\right)/r$
and map it to a predicted gravitational response using the one-parameter empirical curve
$g_{\rm pred}=g_{\rm bar}/\left(1-e^{-\sqrt{g_{\rm bar}/a_0}}\right)$.
In contrast to a fit, we hold parameters fixed at $(\Upsilon_{\rm disk},\Upsilon_{\rm bul})=(0.60,0.70)$ and
$a_0=1.12\times 10^{-10}\,\mathrm{m/s^2}$ (a representative value from an independent RAR fit),
and compare the implied $V_{\rm pred}(r)=\sqrt{g_{\rm pred}(r)\,r}$ against observed $V_{\rm obs}(r)$.
After basic quality cuts ($r>0$, $V_{\rm obs}>0$, finite values), we analyze 3375 radial points from 171 galaxies.
The global RMS residual in $\log_{10} g$ is 0.208 dex (velocity fractional RMS 0.358).
\end{abstract}

\clearpage
\tableofcontents

\chap{Scope and deliverable}
This document is intentionally narrow: it records a \textbf{single} verification task
that can be rerun from public data with short Python code.
The goal is to provide an external ``handle'' on an Enchan-relevant question:
\emph{can a direct baryons-to-dynamics mapping predict full rotation curves with fixed parameters?}
No claim of definitive model selection is made here; we fix definitions, parameters, and outputs.

\chap{Data and definitions}
\subsection*{Data}
We use the SPARC \texttt{Rotmod\_LTG} files (\texttt{[Galaxy]\_rotmod.dat}), which provide:
radius $r$ (kpc), observed velocity $V_{\rm obs}$ (km/s) with uncertainty,
and Newtonian component velocities $V_{\rm gas}$, $V_{\rm disk}$, $V_{\rm bul}$
(km/s), plus surface-brightness metadata.
We apply basic quality cuts ($r>0$, $V_{\rm obs}>0$, finite values).

\subsection*{Accelerations}
We compute
\begin{align}
g_{\rm obs}(r) &= \frac{V_{\rm obs}(r)^2}{r},\\
g_{\rm bar}(r) &= \frac{V_{\rm gas}(r)^2+\Upsilon_{\rm disk}V_{\rm disk}(r)^2+\Upsilon_{\rm bul}V_{\rm bul}(r)^2}{r},
\end{align}
converting $(\mathrm{km/s})^2/\mathrm{kpc}$ to $\mathrm{m/s^2}$.

\subsection*{Fixed mapping and prediction}
We use
\begin{equation}
g_{\rm pred}(g_{\rm bar};a_0)=\frac{g_{\rm bar}}{1-\exp\left(-\sqrt{g_{\rm bar}/a_0}\right)}.
\end{equation}
Parameters are fixed for the entire sample:
\begin{equation}
(\Upsilon_{\rm disk},\Upsilon_{\rm bul})=(0.60,0.70),\qquad a_0=1.12\times 10^{-10}\,\mathrm{m/s^2}.
\end{equation}
Predicted velocity is
\begin{equation}
V_{\rm pred}(r)=\sqrt{g_{\rm pred}(r)\,r},
\end{equation}
where $g_{\rm pred}(r)\equiv g_{\rm pred}(g_{\rm bar}(r);a_0)$.

\chap{Results}
\subsection*{Point-level comparison}
Figure~\ref{fig:vv} compares $V_{\rm pred}$ and $V_{\rm obs}$ across all points.

\begin{figure}[htbp]
\centering
\includegraphics[width=0.82\linewidth]{fig_vpred_vs_vobs.png}
\caption{Point-level comparison of predicted vs observed rotation speed (all galaxies, all radii).}
\label{fig:vv}
\end{figure}

\subsection*{Residual distributions}
We report residuals in $\log_{10} g$:
\begin{equation}
\Delta = \log_{10}(g_{\rm obs}) - \log_{10}(g_{\rm pred}).
\end{equation}
Figure~\ref{fig:hist} shows the residual distribution, and Figure~\ref{fig:galhist}
shows the distribution of per-galaxy RMS residuals.

\begin{figure}[htbp]
\centering
\includegraphics[width=0.90\linewidth]{fig_resid_logg_hist.png}
\caption{Residual distribution in $\log_{10} g$ (global RMS = 0.208 dex).}
\label{fig:hist}
\end{figure}

\begin{figure}[htbp]
\centering
\includegraphics[width=0.90\linewidth]{fig_galaxy_rms_hist.png}
\caption{Distribution of per-galaxy RMS residuals in $\log_{10} g$ (median marker).}
\label{fig:galhist}
\end{figure}

\subsection*{Quick numerical summary}
Table~\ref{tab:sum} summarizes global and per-galaxy metrics.
Note: the per-galaxy reduced $\chi^2$ values use an approximate uncertainty
$\sigma_{\log g}\approx (2\,\sigma_V/V)/\ln(10)$ and ignore radius uncertainty.

\begin{table}[htbp]
\centering
\begin{tabular}{lcc}
\toprule
Metric & Value & Notes\\
\midrule
Galaxies & 171 & after cuts\\
Radial points & 3375 & after cuts\\
Global RMS in $\log_{10} g$ & 0.208 dex & all points\\
Global RMS in fractional $V$ & 0.358 & $(V_{\rm obs}-V_{\rm pred})/V_{\rm obs}$\\
Median per-galaxy reduced $\chi^2$ & 10.37 & approximate $\sigma_{\log g}$\\
Fraction with reduced $\chi^2<2$ & 0.129 & \\
Fraction with reduced $\chi^2<5$ & 0.327 & \\
\bottomrule
\end{tabular}
\caption{Summary of the fixed-parameter rotation-curve prediction test.}
\label{tab:sum}
\end{table}

\chap{Interpretation: geometry vs particles (minimal statement)}
This report establishes one empirical fact from public SPARC mass models:
\textbf{a single fixed mapping $g_{\rm bar}\mapsto g_{\rm pred}$ produces
a nontrivial, sample-wide prediction for full rotation curves without tuning parameters per galaxy.}

\subsection*{What this does and does not show}
This test does \emph{not} falsify particle dark matter, and it does \emph{not} validate any specific
geometric theory by itself. It does provide a reproducible target:
any successful explanation must account for why a tight baryons-to-dynamics mapping
works as well as it does, and where it fails.
The residuals and reduced $\chi^2$ indicate substantial scatter and/or unmodeled systematics,
so this should be read as a baseline fixed-parameter benchmark.

\chap{Reproducibility artifacts}
This report is accompanied by:
\begin{itemize}
\item \texttt{sparc\_vpred\_points\_Yd0p60\_Yb0p70\_a0fixed.csv}: point-level predictions and residuals
\item \texttt{sparc\_vpred\_galaxy\_summary\_Yd0p60\_Yb0p70\_a0fixed.csv}: per-galaxy summary
\item figures: \texttt{fig\_vpred\_vs\_vobs.png}, \texttt{fig\_resid\_logg\_hist.png}, \texttt{fig\_galaxy\_rms\_hist.png}
\end{itemize}

\chap{References}
\begin{itemize}
\item SPARC database (downloads): \url{https://astroweb.case.edu/SPARC/}
\end{itemize}

\end{document}
