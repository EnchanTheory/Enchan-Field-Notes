%==============================================================================
% Enchan BTFR Test Report (SPARC BTFR table)
% Copyright (c) 2025 Mitsuhiro Kobayashi
%
% This work (textual content, PDF) is licensed under a
% Creative Commons Attribution-NonCommercial 4.0 International License (CC BY-NC 4.0).
% To view a copy of this license, visit http://creativecommons.org/licenses/by-nc/4.0/
%
% The LaTeX source code structure itself is available under the MIT License.
%==============================================================================

\documentclass[11pt,a4paper]{article}

%===========================
% Packages (match Enchan Field Notes style)
%===========================
\usepackage[utf8]{inputenc}
\usepackage[T1]{fontenc}
\usepackage{mathptmx}
\usepackage{geometry}
\geometry{margin=1in}
\usepackage{amsmath,amssymb}
\usepackage{bm}
\usepackage{setspace}
\usepackage{hyperref}
\usepackage{booktabs}
\usepackage{graphicx}

\hypersetup{
    colorlinks=true,
    linkcolor=blue,
    citecolor=blue,
    urlcolor=blue
}

%===========================
% Helpers
%===========================
\newcommand{\chap}[1]{\clearpage\section{#1}}

%===========================
% Title / Author
%===========================
\title{\textbf{Enchan BTFR Test Report v0.1}\\[0.5em]
\large One-point-per-galaxy verification from the SPARC BTFR table}
\author{Mitsuhiro Kobayashi\\[0.25em]
Tokyo, Japan\\
\texttt{enchan.theory@gmail.com}}
\date{\today}

\begin{document}

\maketitle
\thispagestyle{empty}

\begin{abstract}
\noindent
This report documents a minimal, reproducible verification of the baryonic Tully--Fisher relation (BTFR)
using a public SPARC BTFR table (\texttt{BTFR\_Lelli2019.mrt}).
We extract one galaxy-level point per object: baryonic mass (stars+gas) and flat rotation velocity $V_{\rm f}$,
and fit a log--log power law of the form $\log_{10} M_{\rm b} = a + b\,\log_{10} V_{\rm f}$.
After basic finite-value cuts and requiring $V_{\rm f}>0$, the sample contains $N=123$ galaxies.
We obtain $a=2.188$, $b=3.748$, and an RMS scatter of 0.235 dex in $\log_{10} M_{\rm b}$
around the best-fit line.
The purpose of this note is to fix a transparent extraction and benchmark fit
that can be rerun from public data with short Python code.
\end{abstract}

\clearpage
\tableofcontents

\chap{Scope and deliverable}
This document is intentionally narrow: it records a \textbf{single} verification task
that can be rerun from public data.
The goal is to create an external, non-spiritual handle on a core ``dark-matter'' symptom:
\textbf{one baryonic number per galaxy predicts one dynamical number per galaxy with small scatter.}
Deliverables:
\begin{itemize}
\item deterministic extraction of $(M_{\rm b},V_{\rm f})$ from the public SPARC BTFR table;
\item a single log--log best-fit line and a compact scatter metric;
\item reproducibility artifacts (CSV + figures + this TeX source).
\end{itemize}
No claim of definitive model selection is made here.

\chap{Data and definitions}
\subsection*{Input table}
We use the SPARC BTFR table distributed as a CDS-style fixed-width \texttt{.mrt} file:
\begin{itemize}
\item \texttt{BTFR\_Lelli2019.mrt}
\end{itemize}
The file contains one row per galaxy, including \texttt{log(Mb)} and the flat rotation velocity \texttt{Vf}
with associated uncertainties.

\subsection*{Quantities}
We define
\begin{equation}
x \equiv \log_{10}(V_{\rm f}/\mathrm{km\,s^{-1}}),\qquad
y \equiv \log_{10}(M_{\rm b}/M_\odot).
\end{equation}
For this benchmark, we take \texttt{log(Mb)} directly from the table and compute $x=\log_{10}(\texttt{Vf})$.

\chap{Fit and metrics}
\subsection*{Line fit}
We fit
\begin{equation}
y = a + b\,x
\end{equation}
using a weighted least squares objective in $y$ with weights $w=1/\sigma_y^2$,
where $\sigma_y$ is the tabulated uncertainty \texttt{e\_log(Mb)} when available.
Errors in $x$ are ignored in this minimal benchmark.

\subsection*{Scatter}
We report the RMS of residuals in $y$:
\begin{equation}
{\rm RMS} = \sqrt{\langle (y - (a+bx))^2\rangle}.
\end{equation}

\chap{Results}
\subsection*{BTFR plot}
Figure~\ref{fig:btfr} shows the BTFR points and the best-fit line.

\begin{figure}[htbp]
\centering
\includegraphics[width=0.88\linewidth]{fig\_btfr\_points.png}
\caption{BTFR from the SPARC table: one point per galaxy in log--log space with best-fit line.}
\label{fig:btfr}
\end{figure}

\subsection*{Residuals}
Figure~\ref{fig:resid} shows residuals about the best-fit line.

\begin{figure}[htbp]
\centering
\includegraphics[width=0.88\linewidth]{fig\_btfr\_residuals.png}
\caption{Residuals in $\log_{10} M_{\rm b}$ around the best-fit BTFR line.}
\label{fig:resid}
\end{figure}

\subsection*{Numeric summary}
\begin{table}[htbp]
\centering
\begin{tabular}{lcc}
\toprule
Metric & Value & Notes\\
\midrule
Galaxies $N$ & 123 & after basic cuts and $V_{\rm f}>0$\\
Intercept $a$ & 2.188 & in $\log_{10} M_{\rm b} = a + b \log_{10} V_{\rm f}$\\
Slope $b$ & 3.748 & log--log slope\\
RMS scatter & 0.235 dex & in $\log_{10} M_{\rm b}$\\
\bottomrule
\end{tabular}
\caption{BTFR fit summary for the selected sample.}
\label{tab:sum}
\end{table}

\chap{Interpretation: geometry vs particles (minimal statement)}
This report establishes one empirical fact from a public SPARC table:
\textbf{a near power-law mapping from baryonic mass to flat rotation velocity is strongly present
at the ``one galaxy = one point'' level.}

\subsection*{Geometric reading (Enchan-style)}
A geometric/field reading treats the galaxy-scale gravitational response as a direct functional of the baryonic configuration,
so a tight galaxy-level mapping $M_{\rm b}\leftrightarrow V_{\rm f}$ is a natural primary object and a compact benchmark.

\subsection*{Particle reading}
A particle dark-matter reading explains $V_{\rm f}$ via baryons plus a halo shaped by assembly history.
A tight BTFR requires an explanation for why baryonic content and halo response co-vary so strongly across galaxies.

\subsection*{What this test does and does not show}
This test does \emph{not} falsify particle dark matter, and it does \emph{not} validate any specific geometric theory by itself.
It fixes a reproducible target statistic (slope and scatter) that any explanation must reproduce.

\chap{Key limitations (explicit)}
\begin{itemize}
\item The analysis uses the table's adopted quantities and systematics; it does not re-derive $M_{\rm b}$ or $V_{\rm f}$ from raw data.
\item The fit ignores uncertainties in $x=\log_{10} V_{\rm f}$ (minimal benchmark).
\item Choice of velocity definition matters; this report uses \texttt{Vf} as provided in the table.
\end{itemize}

\chap{Reproducibility artifacts}
This report is accompanied by:
\begin{itemize}
\item \texttt{btfr\_points\_processed.csv}: extracted galaxy-level values and residuals
\item \texttt{btfr\_fit\_summary.csv}: one-row fit summary (a, b, RMS, N)
\item figures: \texttt{fig\_btfr\_points.png}, \texttt{fig\_btfr\_residuals.png}
\end{itemize}

\chap{References}
\begin{itemize}
\item SPARC database (BTFR tables): \url{https://astroweb.case.edu/SPARC/}
\end{itemize}

\end{document}
