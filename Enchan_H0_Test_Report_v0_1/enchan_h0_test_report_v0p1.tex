%==============================================================================
% Enchan H0 Test Report (TDCOSMO chain export)
% Copyright (c) 2025 Mitsuhiro Kobayashi
%
% This work (textual content, PDF) is licensed under a
% Creative Commons Attribution-NonCommercial 4.0 International License (CC BY-NC 4.0).
% To view a copy of this license, visit http://creativecommons.org/licenses/by-nc/4.0/
%
% The LaTeX source code structure itself is available under the MIT License.
%==============================================================================

\documentclass[11pt,a4paper]{article}

%===========================
% Packages (match Enchan Field Notes style)
%===========================
\usepackage[utf8]{inputenc}
\usepackage[T1]{fontenc}
\usepackage{mathptmx}
\usepackage{geometry}
\geometry{margin=1in}
\usepackage{amsmath,amssymb}
\usepackage{bm}
\usepackage{setspace}
\usepackage{hyperref}
\usepackage{booktabs}
\usepackage{graphicx}

\hypersetup{
    colorlinks=true,
    linkcolor=blue,
    citecolor=blue,
    urlcolor=blue
}

%===========================
% Helpers
%===========================
\newcommand{\chap}[1]{\clearpage\section{#1}}

%===========================
% Title / Author
%===========================
\title{\textbf{Enchan H0 Test Report v0.1}\\[0.5em]
\large Reproducing a public TDCOSMO chain-export posterior for $H_0$}
\author{Mitsuhiro Kobayashi\\[0.25em]
Tokyo, Japan\\
\texttt{enchan.theory@gmail.com}}
\date{\today}

\begin{document}

\maketitle
\thispagestyle{empty}

\begin{abstract}
\noindent
This report documents a minimal, reproducible extraction of the $H_0$ posterior
from a publicly released TDCOSMO chain-export file (\texttt{LambdaCDM1a.h5}).
We load the MCMC samples, select the parameter \texttt{h0} (in km/s/Mpc),
and compute summary statistics (median, 16th/84th percentiles) and a single diagnostic plot.
For the analyzed chain (Dataset: TDCOSMO + Pantheon+; Lens label: RXJ1131-1231),
we obtain $H_0 = 71.64^{+3.89}_{-3.33}$ km/s/Mpc
(16--84\%: 68.30--75.53).
\end{abstract}

\clearpage
\tableofcontents

\chap{Scope and deliverable}
This document is intentionally narrow: it records a \textbf{single} verification task
that can be rerun from public data with short Python code.
The goal is to establish an externally intelligible handle for later work:
\begin{itemize}
\item verify that the publicly released chain-export data reproduce a concrete $H_0$ posterior;
\item fix the extraction steps and outputs (figure + CSV summary) for transparent reuse.
\end{itemize}
No claim of definitive model selection is made here.

\chap{Data and method}
\subsection*{Data}
We analyze the chain-export HDF5 file:
\begin{itemize}
\item file: \texttt{LambdaCDM1a.h5}
\item model label: LambdaCDM (from filename)
\item dataset label: TDCOSMO + Pantheon+
\item lens label in metadata: RXJ1131-1231
\end{itemize}
The file contains an array of samples and a list of parameter names.

\subsection*{Extraction}
We read the dataset \texttt{samples} and the array \texttt{parameters},
locate the column named \texttt{h0}, and compute:
\begin{itemize}
\item median and 16th/84th percentiles of $H_0$;
\item a histogram of the posterior with percentile markers.
\end{itemize}
Additionally, we report $P(H_0<67)$ as a simple reference probability.
This value is \emph{not} a full tension metric and is included only to provide a transparent,
single-number comparison point.

\chap{Results}
\subsection*{Summary statistics}
Table~\ref{tab:h0} summarizes the extracted posterior statistics.

\begin{table}[htbp]
\centering
\begin{tabular}{lcccc}
\toprule
Chain & Median & 16th & 84th & $P(H_0<67)$\\
\midrule
\texttt{LambdaCDM1a} & 71.64 & 68.30 & 75.53 & 0.079\\
\bottomrule
\end{tabular}
\caption{Extracted $H_0$ posterior summary (km/s/Mpc).}
\label{tab:h0}
\end{table}

\subsection*{Posterior shape}
Figure~\ref{fig:h0post} shows the posterior histogram with median and 16th/84th percentile markers.

\begin{figure}[htbp]
\centering
\includegraphics[width=0.88\linewidth]{fig_h0_posterior_LambdaCDM1a.png}
\caption{Posterior for $H_0$ extracted from \texttt{LambdaCDM1a.h5}.}
\label{fig:h0post}
\end{figure}

\chap{Interpretation: geometry vs particles (minimal statement)}
This report establishes one empirical fact about the provided public chain:
\textbf{the extracted posterior has median $H_0$ around 71--72 km/s/Mpc in this configuration.}

\subsection*{Does this test validate a geometric explanation?}
No. The chain samples are produced under standard gravitational lensing modeling assumptions.
This report only verifies and summarizes the released posterior. To test a geometric modification,
one would need a forward model that changes the inferred time-delay distance and then re-fit
(or re-weight) against the data.

\subsection*{How it can still help later Enchan work}
The value and uncertainty reported here provide a reproducible target.
Any proposed mechanism (geometric or particle-based) that aims to address late-vs-early $H_0$
differences can be checked against this posterior using the same extraction interface.

\chap{Reproducibility artifacts}
This report is accompanied by:
\begin{itemize}
\item \texttt{tdcosmo\_h0\_summary\_LambdaCDM1a.csv}: extracted posterior summary
\item \texttt{fig\_h0\_posterior\_LambdaCDM1a.png}: diagnostic posterior plot
\end{itemize}
Reproduction procedure (minimal): load \texttt{LambdaCDM1a.h5}, locate \texttt{h0} in \texttt{parameters},
compute quantiles on the \texttt{samples} column, and export the CSV + histogram figure.

\chap{References}
\begin{itemize}
\item TDCOSMO 2025 public repository (chain export): \url{https://github.com/TDCOSMO/TDCOSMO2025_public}
\end{itemize}

\end{document}
