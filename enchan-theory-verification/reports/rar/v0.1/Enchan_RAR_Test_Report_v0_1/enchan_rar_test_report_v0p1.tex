%==============================================================================
% Enchan RAR Test Report (SPARC / Rotmod_LTG)
% Copyright (c) 2025 Mitsuhiro Kobayashi
%
% This work (textual content, PDF) is licensed under a
% Creative Commons Attribution-NonCommercial 4.0 International License (CC BY-NC 4.0).
% To view a copy of this license, visit http://creativecommons.org/licenses/by-nc/4.0/
%
% The LaTeX source code structure itself is available under the MIT License.
%==============================================================================

\documentclass[11pt,a4paper]{article}

%===========================
% Packages (match Enchan Field Notes style)
%===========================
\usepackage[utf8]{inputenc}
\usepackage[T1]{fontenc}
\usepackage{mathptmx}
\usepackage{geometry}
\geometry{margin=1in}
\usepackage{amsmath,amssymb}
\usepackage{bm}
\usepackage{setspace}
\usepackage{hyperref}
\usepackage{booktabs}
\usepackage{graphicx}

\hypersetup{
    colorlinks=true,
    linkcolor=blue,
    citecolor=blue,
    urlcolor=blue
}

%===========================
% Helpers
%===========================
\newcommand{\chap}[1]{\clearpage\section{#1}}

%===========================
% Title / Author
%===========================
\title{\textbf{Enchan RAR Test Report v0.1}\\[0.5em]
\large Public-data verification of the SPARC Radial Acceleration Relation (RAR)}
\author{Mitsuhiro Kobayashi\\[0.25em]
Tokyo, Japan\\
\texttt{enchan.theory@gmail.com}}
\date{\today}

%===========================
% Document
%===========================
\begin{document}

\maketitle
\thispagestyle{empty}

\begin{abstract}
\noindent
This report documents a minimal, reproducible verification of the
\emph{radial acceleration relation} (RAR) using the publicly released
SPARC rotation-curve decomposition files (\texttt{Rotmod\_LTG}).
We compute the observed centripetal acceleration
$g_{\rm obs}(r)=V_{\rm obs}(r)^2/r$
and the baryonic acceleration proxy
$g_{\rm bar}(r)=\left(V_{\rm gas}^2+\Upsilon_{\rm disk}V_{\rm disk}^2+\Upsilon_{\rm bul}V_{\rm bul}^2\right)/r$,
and fit a one-parameter empirical curve of the form
$g_{\rm obs}=g_{\rm bar}/\left(1-e^{-\sqrt{g_{\rm bar}/a_0}}\right)$.
For a baseline choice $(\Upsilon_{\rm disk},\Upsilon_{\rm bul})=(0.50,0.70)$ we obtain
$a_0\simeq 1.39\times 10^{-10}\,\mathrm{m/s^2}$ and an RMS scatter of $\sim 0.213$ dex.
A one-dimensional scan in $\Upsilon_{\rm disk}$ shows that the weak residual dependence
on disk surface brightness (SBdisk) is minimized near $\Upsilon_{\rm disk}\simeq 0.60$,
without materially changing the overall scatter ($\sim 0.212$ dex).
Compressing the dataset to one point per galaxy (median in $\log g$) preserves a strong
RAR correlation and yields a galaxy-median scatter of $\sim 0.196$ dex around a refit curve.
\end{abstract}

\clearpage
\tableofcontents

\chap{Scope and deliverable}
This document is intentionally narrow: it records a \textbf{single} verification task
that can be rerun from public data with short Python code.
The goal is to establish an externally intelligible handle for later work:
\begin{itemize}
\item the RAR is an empirical regularity that is easy to explain to non-specialists;
\item the same regularity is a natural junction point for contrasting
a geometric interpretation (direct $g_{\rm bar}\rightarrow g_{\rm obs}$ mapping)
with a particle-dark-matter interpretation (visible matter plus a dark component shaped by formation history).
\end{itemize}
No claim of definitive model selection is made here; the report fixes
numbers, definitions, and reproducible artifacts.

\chap{Data and definitions}
\subsection*{Data}
We use the public SPARC rotation-curve decomposition files
\texttt{Rotmod\_LTG} (\texttt{.dat} per galaxy), which tabulate
radius $r$ (kpc), observed velocity $V_{\rm obs}$ (km/s) with uncertainty,
and component velocities $V_{\rm gas}$, $V_{\rm disk}$, and $V_{\rm bul}$,
as well as disk surface brightness SBdisk (when available).
After basic quality cuts ($r>0$, $V_{\rm obs}>0$, finite values) the dataset contains
175 galaxies and 3391 radial points. For analyses that explicitly use SBdisk,
we restrict to points with SBdisk$>0$, yielding 3111 points.

\subsection*{Accelerations}
We compute
\begin{align}
g_{\rm obs}(r) &= \frac{V_{\rm obs}(r)^2}{r},\\
g_{\rm bar}(r) &= \frac{V_{\rm gas}(r)^2+\Upsilon_{\rm disk}V_{\rm disk}(r)^2+\Upsilon_{\rm bul}V_{\rm bul}(r)^2}{r},
\end{align}
converting $(\mathrm{km/s})^2/\mathrm{kpc}$ to $\mathrm{m/s^2}$.
Velocity uncertainties are propagated into an approximate uncertainty for $\log_{10} g_{\rm obs}$
using $\sigma_g/g \approx 2\,\sigma_V/V$ (radius uncertainty ignored for this quick test).

\subsection*{One-parameter reference curve}
We fit an empirical one-parameter curve used widely in the RAR literature:
\begin{equation}
g_{\rm obs}(g_{\rm bar};a_0)=\frac{g_{\rm bar}}{1-\exp\left(-\sqrt{g_{\rm bar}/a_0}\right)}.
\end{equation}
The fit is performed in $\log_{10}$ space with a simple mean-squared residual objective
weighted by the propagated $\sigma_{\log g_{\rm obs}}$.

\chap{Results}
\subsection*{RAR is strongly present in public SPARC data}
Figure~\ref{fig:rar_points} shows the point-level RAR for the SB-clean subset
using the recommended setting $(\Upsilon_{\rm disk},\Upsilon_{\rm bul})=(0.60,0.70)$.
A one-parameter curve provides a compact summary of the trend.

\begin{figure}[htbp]
\centering
\includegraphics[width=0.80\linewidth]{fig_rar_points.png}
\caption{SPARC RAR from \texttt{Rotmod\_LTG} (points) with a one-parameter reference curve.}
\label{fig:rar_points}
\end{figure}

\subsection*{Baseline vs recommended mass-to-light ratio}
Table~\ref{tab:summary} summarizes the baseline and recommended settings.
The overall scatter is stable, while the residual--SBdisk dependence (evaluated on galaxy medians)
is reduced to near zero around $\Upsilon_{\rm disk}\simeq 0.60$.
Figure~\ref{fig:sb_binned} visualizes the binned mean residual versus SBdisk for the two settings.

\begin{table}[htbp]
\centering
\begin{tabular}{lcccc}
\toprule
Setting & $(\Upsilon_{\rm disk},\Upsilon_{\rm bul})$ & $a_0\,[\mathrm{m/s^2}]$ & RMS (dex) & $\rho_{\rm SB}$ (gal medians)\\
\midrule
Baseline & (0.50, 0.70) & $1.39\times 10^{-10}$ & 0.213 & +0.074\\
Recommended & (0.60, 0.70) & $1.12\times 10^{-10}$ & 0.212 & $-0.008$\\
\bottomrule
\end{tabular}
\caption{RAR fit summary (SB-clean subset; 175 galaxies, 3111 points).
$\rho_{\rm SB}$ is Spearman $\rho$ between galaxy-median residual and galaxy-median SBdisk.}
\label{tab:summary}
\end{table}

\begin{figure}[htbp]
\centering
\includegraphics[width=0.86\linewidth]{fig_resid_sb_binned.png}
\caption{Binned mean residual versus SBdisk for baseline and recommended $\Upsilon_{\rm disk}$.}
\label{fig:sb_binned}
\end{figure}

\subsection*{Upsilon scan: SB-dependence is minimized near $\Upsilon_{\rm disk}\simeq 0.60$}
Figure~\ref{fig:scan_rho} shows how the galaxy-median residual--SBdisk correlation changes with $\Upsilon_{\rm disk}$
(keeping $\Upsilon_{\rm bul}=0.70$ fixed), and Figure~\ref{fig:scan_rms} shows the corresponding scatter.
The scan indicates that a small shift in $\Upsilon_{\rm disk}$ can remove the residual SB trend
without degrading the overall scatter.

\begin{figure}[htbp]
\centering
\includegraphics[width=0.86\linewidth]{fig_scan_rho_sb.png}
\caption{Residual--SBdisk dependence (Spearman $\rho$ on galaxy medians) versus $\Upsilon_{\rm disk}$.}
\label{fig:scan_rho}
\end{figure}

\begin{figure}[htbp]
\centering
\includegraphics[width=0.86\linewidth]{fig_scan_rms.png}
\caption{RAR scatter (RMS in dex) versus $\Upsilon_{\rm disk}$.}
\label{fig:scan_rms}
\end{figure}

\subsection*{One point per galaxy}
To avoid overweighting galaxies with many radial points, we compress the dataset to
one point per galaxy using the median of $\log g_{\rm bar}$ and $\log g_{\rm obs}$.
Figure~\ref{fig:rar_gal} shows that the RAR remains clear in this compressed representation.

For the recommended setting, the RMS scatter around the point-level fit is $\sim 0.216$ dex
in the galaxy-median space, and refitting $a_0$ on the galaxy medians yields a reduced scatter
of $\sim 0.196$ dex.

\begin{figure}[htbp]
\centering
\includegraphics[width=0.86\linewidth]{fig_rar_galaxy.png}
\caption{RAR compressed to one point per galaxy (medians in $\log g$).}
\label{fig:rar_gal}
\end{figure}

\chap{Interpretation: geometry vs particles (minimal statement)}
This report establishes one observational fact from public data:
\textbf{the mapping $g_{\rm bar}\mapsto g_{\rm obs}$ is highly regular across many galaxies}
and can be summarized by a simple one-parameter curve with small scatter.

\subsection*{Geometric interpretation}
A geometric interpretation treats the observed gravitational response as a
phenomenological functional of the baryonic configuration, so a tight relation of the form
$g_{\rm obs}=f(g_{\rm bar})$ is a natural primary object.

\subsection*{Particle-dark-matter interpretation}
A particle-dark-matter interpretation constructs $g_{\rm obs}$ as the combined effect of
baryons and a dark halo whose structure depends on assembly history and feedback.
A tight, nearly universal $g_{\rm obs}$--$g_{\rm bar}$ mapping is not automatic and
requires an explanation for why baryons and the halo co-vary so strongly.

\subsection*{What this test does and does not show}
This test does \emph{not} falsify particle dark matter. It \emph{does} provide a clean,
reproducible target: any successful explanation must account for the observed
regularity and its small scatter.

\chap{Reproducibility artifacts}
This report is accompanied by the following files (produced by the Python workflow):
\begin{itemize}
\item \texttt{sparc\_rar\_points\_processed.csv}: point-level accelerations and residuals (recommended setting)
\item \texttt{rar\_Ydisk\_scan\_results.csv}: $\Upsilon_{\rm disk}$ scan summary
\item \texttt{sparc\_rar\_galaxy\_medians\_Yd0p60\_Yb0p70.csv}: one point per galaxy
\item figures: \texttt{fig\_rar\_points.png}, \texttt{fig\_resid\_sb\_binned.png},
\texttt{fig\_scan\_rho\_sb.png}, \texttt{fig\_scan\_rms.png}, \texttt{fig\_rar\_galaxy.png}
\end{itemize}
All quantities are computed from public SPARC \texttt{Rotmod\_LTG} tables with the definitions fixed above.

\chap{References}
\begin{itemize}
\item SPARC database (rotation curves and mass models): \url{http://astroweb.cwru.edu/SPARC/}
\item McGaugh, Lelli, Schombert (2016), ``The Radial Acceleration Relation in Rotationally Supported Galaxies'', arXiv:1609.05917.
\end{itemize}

\end{document}
