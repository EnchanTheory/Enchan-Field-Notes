%==============================================================================
% Enchan Field Notes (Theoretical Framework)
% Version: \version
% Main file - loads sections from ./sections/
%==============================================================================

\documentclass[11pt,a4paper]{article}

%===========================
% Packages
%===========================
\usepackage[utf8]{inputenc}
\usepackage[T1]{fontenc}
\usepackage{mathptmx}
\usepackage{geometry}
\geometry{margin=1in}
\usepackage{amsmath,amssymb,amsthm}
\usepackage{bm}
\usepackage{setspace}
\usepackage{hyperref}
\usepackage{booktabs}
\usepackage{graphicx}

\hypersetup{
    colorlinks=true,
    linkcolor=blue,
    citecolor=blue,
    urlcolor=blue,
    pdftitle={Enchan Field Notes},
    pdfauthor={Mitsuhiro Kobayashi},
    pdfsubject={Scalar Field Description of Galaxy-Scale Regularities},
    pdfkeywords={galaxy dynamics, effective theory}
}

%===========================
% Helpers
%===========================
\newcommand{\chap}[1]{\clearpage\section{#1}}
\newcommand{\EnchanS}{S}

\pdfstringdefDisableCommands{%
  \def\EnchanS{S}%
  \def\bm#1{#1}%
  \def\mathrm#1{#1}%
  \def\mathcal#1{#1}%
}

%===========================
% Version
%===========================
\newcommand{\version}{v0.4.5}

%===========================
% Title / Author
%===========================
\title{\textbf{Enchan Field Notes \version}\\[0.5em]
\large A Scalar-Field Description of Galaxy-Scale Acceleration Regularities}

\author{Mitsuhiro Kobayashi\\
Tokyo, Japan}

\date{}

%===========================
% Document
%===========================
\begin{document}

\maketitle
\thispagestyle{empty}

%===========================
% Abstract
%===========================
\begin{abstract}
\noindent
We present a compact theoretical framework that organizes several well-known
galaxy-scale regularities into an effective scalar-field description.
A dimensionless field $\EnchanS$ is introduced as a proxy for gravitational
time-delay effects in the weak-field regime.

The framework is constructed to reproduce the Radial Acceleration Relation (RAR)
and the Baryonic Tully--Fisher Relation (BTFR) through defect-like configurations
of $\EnchanS$ anchored by baryonic structure.
Environmental suppression in deep baryonic potentials is incorporated as an
effective modulation of the acceleration scale, calibrated directly on
observational data.

The purpose of these notes is not to modify General Relativity, but to provide
a minimal and internally consistent parametrization of observed phenomenology
within a controlled effective description.
\end{abstract}

%===========================
% Scope
%===========================
\vspace{1em}
\hrule
\section*{Scope and Domain of Applicability}
This document is restricted to the weak-field, galaxy-scale regime in which
empirical acceleration regularities are defined.
No claims are made regarding strong-field gravity, cosmology at early times,
or precision Solar-System tests.
\hrule
\vspace{1em}

\clearpage
\tableofcontents
\clearpage

%===========================
% Content
%===========================
%==============================================================================
% Section 1: Overview and Paradigm Shift
%==============================================================================

\chap{Overview and Paradigm Shift}

The Enchan framework starts from a radical but physically motivated premise: spacetime geometry is not fundamental. Instead, the spatial structure we experience is a stabilized configuration of underlying microscopic fluctuations, and the phenomenon we call "gravity" is the thermodynamic cost of maintaining this order against a cosmic flow.

\textbf{In this \version update}, the framework undergoes a significant paradigm shift. Based on the null results of macroscopic resonance searches (v0.2.x) and the "scale barrier" inherent in generating gravitational waves, we pivot from a model of "Resonant Space" to a model of \textbf{"Topological Defects in an Embedded Spacetime."}

\subsection{The Pivot: From Resonance to Effective Time Dilation}

Previous versions (v0.2.x) hypothesized that spacetime behaves like a resonant membrane that can be vibrated by macroscopic rotation. However, observational constraints and theoretical reassessments have led to the following conclusions:

\begin{itemize}
    \item \textbf{Negative Result on Resonance:} The universe does not "ring" like a bell at human scales. The fabric of space is too rigid ($\sigma_{\text{vac}}$ is too large) to be driven into resonance by laboratory-scale mass-energy.

    \item \textbf{The Embedding Hypothesis (Metaphor: "Inception"):} Instead of vibrating space, we model the universe as a 3-brane embedded in a higher-dimensional bulk geometry. In this view, what we perceive as \textbf{proper time} is interpreted as the velocity of motion through the bulk (geodesic flow).

    \item \textbf{Gravity as Anchoring (Drag):} Mass and rotation do not merely "curve" space; they act as \textbf{obstacles} to this bulk flow. This creates a local "drag" or deceleration of time, which we perceive as a gravitational potential.
\end{itemize}

Consequently, the central scalar field $S(x)$ is redefined in this version not as a displacement field, but as a \textbf{dimensionless time-dilation field}.

\subsection{Core Pillars of \version}

This version of the Field Notes is built upon three theoretical pillars:

\begin{enumerate}
    \item \textbf{Geometric Dark Matter (Spacetime Wrinkles):}
    Dark matter is not a particle species. It is the energy density associated with persistent topological defects ($\nabla S$) in the time-dilation field. These "wrinkles" are remnants of the primordial geometry.

    \item \textbf{The Anchor Condition:}
    Baryonic matter couples to these defects. We introduce the "Anchor Condition," which postulates that defects are pinned by baryonic mass at a characteristic scale $r_c$. This condition naturally recovers the phenomenological success of MOND (Modified Newtonian Dynamics) without modifying the laws of inertia.

    \item \textbf{Unified Acceleration Scale ($a_0$):}
    We propose that the critical acceleration scale $a_0 \approx 1.2 \times 10^{-10}$ m/s$^2$ emerges as a derived parameter determined by the ratio of the vacuum stiffness ($\sigma_{\text{vac}}$) to the defect core energy.
\end{enumerate}

\subsection{Structure of This Document}

This document serves as the technical ledger for the Enchan framework.

\begin{itemize}
    \item \textbf{Chapter 2} outlines the conceptual cosmology of the "Inception World," defining the relationship between the bulk, the singularity, and our internal time.
    \item \textbf{Chapter 3} discusses the role of rotation and "Frame Dragging" as the physical mechanism for anchoring time, referencing recent observational evidence of Lense-Thirring effects.
    \item \textbf{Chapter 4} defines the minimal field model and the Lagrangian density.
    \item \textbf{Chapter 5} explores the asymptotic structure and galactic dynamics arising from the field equations.
    \item \textbf{Chapter 6} presents the \textbf{Theoretical Consistency of $a_0$}, demonstrating how the logarithmic defect solution unifies the Baryonic Tully-Fisher Relation (BTFR) and the Radial Acceleration Relation (RAR).
    \item \textbf{Chapter 7} discusses observational signatures and the roadmap for testing the hypothesis against cosmological data.
\end{itemize}

\subsection*{Scope and Disclaimer}
This framework is proposed as a "source code" layer beneath the effective theories of the Standard Model and General Relativity. It is designed to explain the "Dark Sector" (Dark Matter/Energy) and the origin of inertia. Standard physics is assumed to remain valid in regimes where the gradient of the Enchan field ($\nabla S$) is negligible.
%==============================================================================
% Section 2: Conceptual Setup and Effective Variables
%==============================================================================

\chap{Conceptual Setup and Effective Variables}

This chapter introduces the minimal conceptual setup used throughout these notes.
We work with a dimensionless scalar field $S(x)$ as an effective proxy variable
in the weak-field regime and define the derived quantities used in later chapters.

\subsection{Field variable}

The scalar field $S(x)$ is taken to be dimensionless.
In the intended regime of applicability, spatial gradients of $S$ are used to
parametrize an effective gravitational response, summarized by
\begin{equation}
g_{\rm tot} \equiv c^2 \nabla S .
\end{equation}
This relation is used as a working definition for the phenomenological mapping
between the field variable and the inferred acceleration field in galaxies.

\subsection{Derived quantities}

Two derived quantities are used repeatedly:

\begin{enumerate}
    \item \textbf{Gradient magnitude.}
    We denote the magnitude of the spatial gradient by
    \begin{equation}
        D(x) \equiv \|\nabla S\| .
    \end{equation}
    In the effective description, $D$ serves as a compact measure of spatial
    variation in $S$.

    \item \textbf{Kinetic invariant.}
    For covariant expressions we define
    \begin{equation}
        \mathcal{X} \equiv \frac{\nabla_\mu S\,\nabla^\mu S}{a_0^2},
    \end{equation}
    where $a_0$ is the characteristic acceleration scale introduced later.
    The function $\mu(\mathcal{X})$ used in the field equation is chosen so that
    the quasi-static limit reproduces the empirical interpolation employed in
    the observational benchmarks.
\end{enumerate}

\subsection{Effective source structure}

The galaxy-scale description uses baryonic structure as the primary source input.
We denote the baryonic mass density by $\rho_{\rm b}$ and adopt an anchoring ansatz
in which baryonic distributions select characteristic scales for spatial variations
of $S$.
Environmental modulation is incorporated through an effective suppression of the
acceleration scale in deep baryonic potentials, using a potential-depth proxy.

\subsection{Notation and conventions}

Unless stated otherwise, we work in the weak-field, nonrelativistic limit when
connecting the field model to observational quantities.
The field $S$ is treated as an effective variable; no assumptions are made here
about a unique fundamental completion beyond the regime studied in these notes.

%==============================================================================
% Section 3: Rotation and Kinematic Context
%==============================================================================

\chap{Rotation and Kinematic Context}

This chapter summarizes the role of rotation in the observational and theoretical
context relevant for galaxy dynamics. Rotation curves provide direct access to the
radial acceleration field inferred from circular motion, and therefore serve as a
primary empirical interface for the effective description developed in later chapters.

\subsection{Rotation curves as acceleration data}

For an axisymmetric system with an observed circular velocity profile $v(r)$,
the centripetal acceleration inferred from kinematics is
\begin{equation}
g_{\rm obs}(r) \equiv \frac{v^2(r)}{r}.
\end{equation}
The baryonic contribution is computed from the luminous components under standard
mass-to-light and geometric assumptions, yielding a baryonic acceleration proxy
$g_{\rm bar}(r)$.
The empirical relations discussed in these notes are formulated in terms of the
pair $(g_{\rm obs}, g_{\rm bar})$ across radii and across galaxies.

\subsection{Relativistic rotation effects (context only)}

In General Relativity, rotating mass distributions admit frame-dependent effects
associated with the gravitomagnetic sector of the metric (often referred to as
frame dragging in the weak-field limit).
In these notes, such effects are mentioned only as contextual motivation for why
rotation can be a relevant structural feature of gravitational systems.
No strong-field modeling or device-level interpretation is assumed or required.

\subsection{Connection to the effective description}

The scalar-field parametrization introduced in Chapters 2 and 4 is connected to
galaxy kinematics through the effective acceleration field.
In particular, the working definition
\begin{equation}
g_{\rm tot} \equiv c^2 \nabla S
\end{equation}
is used to map the scalar field variable to an acceleration response in the weak-field,
quasi-static regime.
The remainder of this document focuses on reproducing the observed regularities
in the kinematic data using baryonic inputs and an effective field equation.

\subsection*{Scope}

This chapter provides kinematic definitions and limited theoretical context.
It introduces no new assumptions beyond those stated in Chapters 1--2.

%==============================================================================
% Section 4: Minimal Enchan Field Model
%==============================================================================

\chap{Minimal Enchan Field Model}

In this chapter, we formalize the Enchan framework as a scalar field theory. We construct a Lagrangian density that describes how the time-dilation field $S$ propagates through the "rigid" vacuum of the Inception World, driven by primordial fluctuations and stabilized by a potential.

\subsection{Field Variables and Dimensional Analysis (\version)}

To ensure physical consistency with the derivation of $a_0$ (Chapter 5), we explicitly fix the dimensions of the fields and constants. We adopt the standard SI units where the Lagrangian density $\mathcal{L}$ has dimensions of Energy Density $[J/m^3]$.

\begin{itemize}
    \item \textbf{Enchan Field $S(x)$ [Dimensionless]:}
    The order parameter representing the local time dilation. $S=0$ corresponds to the asymptotic vacuum (free fall).

    \item \textbf{Vacuum Stiffness $\sigma_{\text{vac}}$ [Force, N]:}
    A fundamental coupling constant representing the "rigidity" of the spacetime fabric against time dilation. It converts the dimensionless gradient $(\nabla S)^2$ into an energy density.
    
    \item \textbf{Fluctuation Source $F(x)$ [Energy Density, $J/m^3$]:}
    The external driving force from the Upper Universe (e.g., the roughness of the singularity).

    \item \textbf{Stabilizing Potential $V(S)$ [Energy Density, $J/m^3$]:}
    The self-interaction energy of the field, responsible for forming topological defects.
\end{itemize}

\subsection{The Lagrangian Density}

We propose the following effective Lagrangian density for the Enchan field:

\begin{equation}
    \boxed{
    \mathcal{L} \;=\;
    \frac{\sigma_{\text{vac}}}{2} (\partial_\mu S)(\partial^\mu S)
    \;+\; F S
    \;-\; V(S)
    }
    \label{eq:Lagrangian}
\end{equation}

\paragraph{Physical Interpretation of Terms:}
\begin{enumerate}
    \item \textbf{Kinetic Term ($\frac{\sigma_{\text{vac}}}{2} (\partial S)^2$):}
    Represents the elastic energy cost of having "uneven time." Because $\sigma_{\text{vac}}$ is large (spacetime is rigid), gradients in $S$ are energetically costly. This term eventually manifests as the "Geometric Dark Matter" energy density.
    
    \item \textbf{Source Term ($F S$):}
    Represents the coupling to the primordial fluctuations. A non-zero $F$ drives $S$ away from zero, creating initial seeds for structures.
    
    \item \textbf{Potential Term ($-V(S)$):}
    Represents the "Anchoring" energy. As derived in Chapter 5, the requirement for logarithmic defects imposes a specific form on this potential: $V(S) \sim e^{-2S}$.
\end{enumerate}

\subsection{Equation of Motion}

Varying the action associated with Eq.~\eqref{eq:Lagrangian} with respect to $S$ yields the Euler-Lagrange equation:

\begin{equation}
    \partial_\mu \left( \frac{\partial \mathcal{L}}{\partial (\partial_\mu S)} \right) - \frac{\partial \mathcal{L}}{\partial S} = 0
\end{equation}

Substituting our specific Lagrangian:

\begin{equation}
    \sigma_{\text{vac}} \partial_\mu \partial^\mu S \;=\; F \;-\; \frac{dV}{dS}
\end{equation}

Or, using the d'Alembertian operator $\square$:

\begin{equation}
    \boxed{
    \sigma_{\text{vac}} \square S \;=\; F \;-\; V'(S)
    }
    \label{eq:FieldEq}
\end{equation}

This is the fundamental field equation of Enchan \version. It states that:
\begin{quote}
    \textit{"The curvature of time ($\square S$) is driven by the primordial roughness ($F$) and resisted by the anchoring potential ($V'$), scaled by the stiffness of the vacuum ($\sigma_{\text{vac}}$)."}
\end{quote}

\subsection{Static Limit and Dark Matter Halo}

In the static limit (galactic scales) and far from the core (where $F \approx 0$), the equation simplifies to:
\begin{equation}
    \sigma_{\text{vac}} \nabla^2 S \approx -V'(S).
\end{equation}
This is the equation we solved in Chapter 5 to derive the $a_0$ scale. The resulting gradient energy density is:
\begin{equation}
    \rho_{\text{DM}} \approx \frac{\sigma_{\text{vac}}}{2} (\nabla S)^2.
\end{equation}
Since the defect solution gives $|\nabla S| \propto 1/r$, the energy density scales as:
\begin{equation}
    \rho_{\text{DM}} \propto \frac{1}{r^2}.
\end{equation}
This $1/r^2$ profile is exactly what is required to produce flat rotation curves in galaxies, identifying the "Enchan Field Gradient" as the physical substance of Dark Matter halos.

\subsection{Relation to General Relativity (Effective Metric)}

Although we treat $S$ as a scalar field on a background, it effectively modifies the metric experienced by matter. The line element in the presence of an Enchan field can be approximated (in the weak field limit) as:
\begin{equation}
    ds^2 \approx -c^2 e^{-2S} dt^2 + e^{2S} d\mathbf{x}^2.
\end{equation}
This confirms that $S$ acts as a conformal factor or a time-dilation potential. Matter follows the geodesics of this effective metric, which leads to the observed "extra gravity" without requiring additional mass.
%==============================================================================
% Section 5: Asymptotic Structure and Galactic Dynamics
%==============================================================================

\chap{Asymptotic Structure and Galactic Dynamics}

In this chapter, we explore the physical consequences of the Enchan field equations at galactic scales. We show that the "Inception" vacuum naturally supports long-range topological defects, which manifest observationally as "Dark Matter Halos" with a density profile falling as $1/r^2$, leading inevitably to flat rotation curves.

\subsection{The Static Limit}

For a galaxy in equilibrium, we consider the static limit of the field equation derived in Chapter 4:
\begin{equation}
    \sigma_{\text{vac}} \nabla^2 S \approx -V'(S).
\end{equation}
We are interested in the behavior of the field far from the galactic center ($r \gg r_c$), where the direct influence of baryonic matter is negligible, and the dynamics are dominated by the self-interaction of the field (the vacuum structure).

\subsection{The Logarithmic Wrinkle (Topological Defect)}

As derived in Chapter 6, the stability condition for a topological defect in this framework requires the field to behave logarithmically at large distances:
\begin{equation}
    S(r) \sim \eta_S \ln\left(\frac{r}{r_c}\right).
\end{equation}

\paragraph{Physical Interpretation:}
Standard Newtonian gravity relies on the Poisson equation $\nabla^2 \Phi = 4\pi G \rho$, where a point source creates a potential $\Phi \sim 1/r$ and a force $g \sim 1/r^2$.
However, the Enchan field $S$ represents a \textbf{"Wrinkle" or "Tear" in the time flow}. Unlike a point mass that fades away, a topological defect carries a global constraint. The stress in the vacuum ($\nabla S$) cannot relax faster than $1/r$ due to the topology of the Inception World.

\begin{itemize}
    \item \textbf{Gradient Profile:} The "drag" on time decays slowly:
    \begin{equation}
        |\nabla S| = \frac{\eta_S}{r}.
    \end{equation}
    \item \textbf{Comparison:} This is a much slower decay than Newtonian gravity ($1/r^2$). This "long-range drag" is the origin of the mass discrepancy in the outer regions of galaxies.
\end{itemize}

\subsection{Geometric Dark Matter Halo}

We identified the energy density of the dark sector with the gradient energy of the field:
\begin{equation}
    \rho_{\text{DM}} \equiv \frac{\sigma_{\text{vac}}}{2} (\nabla S)^2.
\end{equation}
Substituting the gradient profile $|\nabla S| = \eta_S / r$:
\begin{equation}
    \rho_{\text{DM}}(r) = \frac{\sigma_{\text{vac}}}{2} \left( \frac{\eta_S}{r} \right)^2 = \frac{\sigma_{\text{vac}} \eta_S^2}{2} \frac{1}{r^2}.
\end{equation}

\paragraph{The 1/$r^2$ Profile:}
This result is profound. Without introducing any new particles, the field theory predicts a "Halo" of energy density that falls off as $1/r^2$.
This is mathematically identical to the density profile of a \textbf{Singular Isothermal Sphere}, which is known to produce perfectly flat rotation curves.

\begin{quote}
    \textbf{Conclusion:} The "Dark Matter Halo" is not a cloud of invisible particles. It is the \textbf{elastic stress energy} of the vacuum stored in the topological wrinkle surrounding the galaxy.
\end{quote}

\subsection{Flat Rotation Curves}

The orbital velocity $v(r)$ of a test particle in this effective halo is determined by the enclosed mass $M_{\text{DM}}(r)$.
\begin{equation}
    M_{\text{DM}}(r) = \int_0^r 4\pi x^2 \rho_{\text{DM}}(x) \, dx = \int_0^r 4\pi x^2 \left( \frac{\mathcal{A}}{x^2} \right) dx = 4\pi \mathcal{A} r,
\end{equation}
where $\mathcal{A} = \sigma_{\text{vac}} \eta_S^2 / 2$ is a constant related to the tension of the wrinkle.
The circular velocity is then:
\begin{equation}
    v^2(r) = \frac{G M_{\text{DM}}(r)}{r} = \frac{G (4\pi \mathcal{A} r)}{r} = 4\pi G \mathcal{A} = \text{const}.
\end{equation}

This demonstrates that \textbf{flat rotation curves are a generic prediction} of the Inception World hypothesis. They are not anomalies; they are the expected behavior of gravity in a universe containing topological defects.

\subsection{The Anchor Effect and Baryonic Coupling}

Why do these wrinkles form around galaxies?
According to the \textbf{Anchor Condition} (Chapter 6), baryonic mass acts as the pinning center for these defects.
\begin{itemize}
    \item A galaxy is not just a collection of stars; it is a "knot" in the time-flow.
    \item The baryonic mass $M_b$ determines the core radius $r_c$ and the amplitude of the wrinkle.
    \item This tight coupling explains the \textbf{Renzo's Rule} and the \textbf{Radial Acceleration Relation (RAR)}: the detailed features of the baryonic distribution are imprinted on the rotation curve because the baryons are physically shaping the defect.
\end{itemize}

\subsection{Cosmological Cutoff}

The logarithmic potential ($S \sim \ln r$) and the linear mass growth ($M \sim r$) cannot continue indefinitely, as the total energy would diverge.
In the Enchan cosmology, this divergence is naturally cut off by:
\begin{enumerate}
    \item \textbf{Neighboring Defects:} The wrinkle of one galaxy eventually merges with the wrinkles of its neighbors (cluster scale).
    \item \textbf{Cosmic Horizon:} The finite size of the observable universe (or the Inception bubble) provides a hard cutoff.
\end{enumerate}
This suggests that "Dark Matter" effects are most prominent at galactic to cluster scales, where the $1/r$ gradient can dominate, but must saturate at cosmological scales.
%===========================
% Chapter 6: Derivation / Acceleration Scale
%===========================

\chap{Derivation: the acceleration scale}

\subsection{Purpose and scope}
This chapter provides a \textbf{model-facing derivation ledger} for the acceleration scale $a_0$.
The goal is not to claim completion, but to fix:
(i) the assumptions used in the derivation,
(ii) the dimensional bookkeeping, and
(iii) a concrete, testable dependence that can be confronted with public data.

\subsection{Assumptions used in this chapter}
We work under the following controlled assumptions:
\begin{itemize}
\item \textbf{Static limit:} $\partial_t S \simeq 0$ on galactic scales.
\item \textbf{Spherical defect ansatz:} the dominant large-scale defect profile is treated as effectively spherical.
\item \textbf{Anchor condition (ansatz):} the defect core energy density is tied to a baryonic \emph{surface}-density proxy evaluated at a characteristic anchor radius $r=r_c$.
\end{itemize}
These assumptions are explicitly flagged so that later revisions can replace them with a more fundamental construction.

\subsection{Field normalization and dimensions}
We adopt the \version convention that the Enchan field $S$ is \textbf{dimensionless} and is interpreted as an effective time-dilation factor.
The Lagrangian density is written schematically as
\begin{equation}
\mathcal{L} = \frac{\sigma_{\rm vac}}{2}(\partial_\mu S)(\partial^\mu S) - V(S),
\end{equation}
where $\sigma_{\rm vac}$ is a universal stiffness parameter (vacuum ``rigidity'') and $V(S)$ is an effective potential encoding anchoring and defect-core energetics.
This chapter uses $\sigma_{\rm vac}$ and a defect-core energy density scale $V_0$ to define the length and acceleration scales below.

\subsection{Defect length scale}
We introduce a characteristic core/anchor length scale $\ell_c$ via
\begin{equation}
\ell_c \equiv \sqrt{\frac{\sigma_{\rm vac}}{2V_0}},
\label{eq:lc_def}
\end{equation}
where $V_0$ has units of energy density and represents the effective defect-core energy density associated with the anchoring region.
This definition is used as a convenient parameterization of the ``stiffness versus core energy'' balance.

\subsection{Defect profile ansatz}
At radii outside the core, we take a logarithmic defect profile,
\begin{equation}
S(r) \simeq S_0 + \eta_S \ln\!\left(\frac{r}{\ell_c}\right),
\label{eq:S_log_profile}
\end{equation}
where $\eta_S$ is a dimensionless defect-strength parameter and $S_0$ is a gauge constant.
This log profile is the minimal form that yields an asymptotic gradient $|\nabla S|\propto 1/r$.

\subsection{The Surface-Density Anchor: A Pressure-Matching Condition}
\label{sec:anchor_derivation}

We now formulate the local condition that determines the topological defect strength from the baryonic distribution.
A key design choice is to anchor the defect using a \textbf{surface-density} proxy rather than a volume density, thereby avoiding the enormous dynamic range of $\rho_{\rm b}$ across galactic scales and focusing on the relevant dynamical quantity for disk systems.

Dimensional analysis provides a robust guide for this coupling.
We introduce an \emph{effective defect stress scale}, $P_{\rm def}$, defined as:
\begin{equation}
P_{\rm def}(r) \equiv \frac{V_0(r)}{c^2 \eta_S}.
\end{equation}
By construction, $P_{\rm def}$ has the dimensions of pressure ($[M][L]^{-1}[T]^{-2}$).
On the baryonic side, consider a self-gravitating sheet with surface mass density $\Sigma_{\rm b}$. The characteristic gravitational pressure (or vertical stress) within the sheet scales as $P_{\rm sg} \sim G \Sigma_{\rm b}^2$, up to an order-unity geometric factor.

We therefore postulate that the defect's effective pressure scale matches the baryonic gravitational pressure at the anchor radius $r_c$.
The \textit{minimal anchor ansatz} is then the condition of pressure equilibrium:
\begin{equation}
P_{\rm def}(r_c) \simeq \frac{1}{\alpha_c} P_{\rm sg}(r_c)
\quad \Longrightarrow \quad
\frac{V_0(r_c)}{c^2 \eta_S} \simeq \frac{1}{\alpha_c} \, G \, \Sigma_{\rm b}(r_c)^2,
\label{eq:V0_anchor}
\end{equation}
where $\alpha_c$ is a dimensionless coupling efficiency factor of order unity.
This relation implies that the vacuum texture is locally "pinned" or regulated by the self-gravity of the baryonic sheet.
Equation (\ref{eq:V0_anchor}) recovers the scaling dependence used in Eq.~(19), but frames it as a leading-order effective field theory (EFT) description consistent with pressure dimensions, rather than an arbitrary choice.

\subsubsection{Observational Proxy for \texorpdfstring{$\Sigma_{\rm b}$}{Sigma\_b}}
For practical confrontation with SPARC data, we treat $\Sigma_{\rm b}(r_c)$ not as a theoretically exact quantity, but as an \textbf{observational proxy} derived from surface photometry.
Assuming the disk dominates the potential at the anchor radius, we adopt:
\begin{equation}
\Sigma_{\rm b}(r_c) \approx \Upsilon_* I_{\rm disk}(r_c) + \Sigma_{\rm gas}(r_c),
\label{eq:sigma_proxy}
\end{equation}
where $\Upsilon_*$ is the stellar mass-to-light ratio and $I_{\rm disk}$ is the observed luminosity profile.

We acknowledge that this proxy carries systematic uncertainties (e.g., from $\Upsilon_*$ variations, geometry, or bulge contamination).
However, the primary goal of the differential prediction test (Test C1) is to verify the \textit{sign and power-law index} of the dependence implied by Eq.~(\ref{eq:V0_anchor})—specifically that the relevant acceleration scale scales with surface density—rather than to strictly constrain the normalization constants.
Any global systematic offsets in the proxy are absorbed into the effective coupling factor $\alpha_c$.

\subsection{Derived acceleration scale}
We define the effective acceleration scale $a_0$ by combining the defect profile parameter $\eta_S$ with the characteristic length $\ell_c$:
\begin{equation}
a_0 \equiv \frac{c^2\,\eta_S}{\ell_c}
= c^2\,\eta_S \sqrt{\frac{2V_0}{\sigma_{\rm vac}}}.
\label{eq:a0_def}
\end{equation}
Using the anchor scaling in Eq.~\eqref{eq:V0_anchor}, this implies the parametric dependence
\begin{equation}
a_0 \propto \eta_S\,\Sigma_{\rm b}(r_c)\,\sqrt{\frac{G}{\sigma_{\rm vac}}}\,,
\label{eq:a0_scaling}
\end{equation}
i.e.\ $a_0$ is not assumed fundamental, but is controlled by vacuum stiffness and an anchored surface-density proxy (up to dimensionless factors).

\subsection{Testable prediction}
Because Eq.~\eqref{eq:a0_scaling} links $a_0$ to a baryonic surface-density proxy, this framework makes a direct, falsifiable statement:
\begin{quote}
\textbf{Prediction:} the per-galaxy acceleration scale inferred from galaxy-level relations should exhibit a \emph{weak} correlation with surface-brightness indicators (or related $\Sigma_{\rm b}$ proxies), rather than being perfectly universal.
\end{quote}
If a near-universal $a_0$ continues to describe diverse galaxies, this must correspond to either
(i) weak variation of the relevant proxy, or
(ii) self-regulation (e.g.\ compensating variation in $\eta_S$ and/or $\alpha_c$).
Either outcome is informative and can be tested.

\subsection{Connection to the empirical benchmarks}
Sec.~\ref{sec:obs_btfr} and Sec.~\ref{sec:obs_rar} (benchmarks) summarize the observed emergence of an acceleration scale $\sim 10^{-10}\,\mathrm{m/s^2}$ from public data.
This chapter provides a \textbf{theory-side parametrization} of where such a scale could enter the model, and fixes the dependencies that must be checked against the benchmark outputs.

%===========================================================
% \version Addendum: Effective Field Equation & Transition
%===========================================================
\subsection{Effective field equation and a candidate transition function}
\label{sec:transition_function}

This section provides an \emph{effective} (approximate) derivation of the transition
between the low-acceleration and high-acceleration regimes. The goal is to connect
the benchmark closure used in v0.3.x to a field-equation form suitable for \version.

\subsubsection{Quasi-static limit and the source term}
We assume a quasi-static, weak-field regime on galactic scales, and identify the
effective gravitational response with the gradient of the dimensionless time-dilation field:
\begin{equation}
\bm{g}_{\rm tot} \equiv c^2 \nabla S .
\end{equation}
Because matter follows the physical metric that depends on $S$ (time dilation),
the matter sector contributes an effective source for $S$ in the non-relativistic limit.
We therefore adopt the following Poisson-like \emph{effective} equation:
\begin{equation}
\nabla \cdot \left[ \mu\!\left(\frac{g_{\rm tot}}{a_0}\right)\, \bm{g}_{\rm tot} \right]
= 4\pi G \rho_{\rm bar}.
\label{eq:mu_poisson}
\end{equation}
Equation~\eqref{eq:mu_poisson} should be read as an effective macroscopic description
of the vacuum texture response. A microscopic derivation from a specific defect model
is deferred to future work.

\subsubsection{Transition function consistent with the quadrature closure}
In spherical symmetry, Eq.~\eqref{eq:mu_poisson} reduces to the algebraic relation
\begin{equation}
\mu\!\left(\frac{g_{\rm tot}}{a_0}\right)\, g_{\rm tot} = g_{\rm bar},
\qquad
g_{\rm bar}(r)\equiv \frac{G M_{\rm bar}(<r)}{r^2}.
\label{eq:mu_relation}
\end{equation}
The v0.3.x benchmarks use the ``Enchan quadrature'' closure
\begin{equation}
g_{\rm tot}=\sqrt{g_{\rm bar}^2+a_0 g_{\rm bar}}
= g_{\rm bar}\,\nu\!\left(\frac{g_{\rm bar}}{a_0}\right),
\qquad
\nu(y)=\sqrt{1+\frac{1}{y}} .
\label{eq:nu_quadrature}
\end{equation}
We define a candidate $\mu$-function that reproduces this closure:
\begin{equation}
\mu\!\left(x\right)=\frac{g_{\rm bar}}{g_{\rm tot}}
=\frac{\sqrt{1+4x^2}-1}{2x},
\qquad x\equiv \frac{g_{\rm tot}}{a_0}.
\label{eq:mu_quadrature}
\end{equation}
Equations~\eqref{eq:nu_quadrature}--\eqref{eq:mu_quadrature} define a transition function that is \emph{by construction} consistent with the benchmark closure.
The theory-side task for \version+ is to derive (or approximate) this response from an explicit
defect profile and anchoring mechanism, rather than postulating it.

\subsubsection{Why the modification ``hides'' in strong-gravity environments}
The limits of Eq.~\eqref{eq:nu_quadrature} are:
\begin{itemize}
\item \textbf{Low-acceleration (deep) regime $g_{\rm bar}\ll a_0$:}
\begin{equation}
g_{\rm tot}\simeq \sqrt{a_0 g_{\rm bar}},
\end{equation}
which yields asymptotically flat rotation curves ($v^2\simeq \sqrt{G M_{\rm bar} a_0}$).

\item \textbf{High-acceleration regime $g_{\rm bar}\gg a_0$:}
\begin{equation}
g_{\rm tot}= g_{\rm bar}\sqrt{1+\frac{a_0}{g_{\rm bar}}}
\simeq g_{\rm bar}\left(1+\frac{a_0}{2g_{\rm bar}}\right).
\end{equation}
For Solar-System/terrestrial accelerations ($g_{\rm bar}\gg a_0$), the fractional correction $\Delta g/g_{\rm bar}\sim a_0/(2g_{\rm bar})$ becomes negligibly small, making the deviation effectively unobservable in practice.
\end{itemize}

\subsubsection{Connection to the RAR benchmarks}
Using $\nu(y)=\sqrt{1+1/y}$, the closure can be written as
\begin{equation}
g_{\rm tot}=g_{\rm bar}\,\nu\!\left(\frac{g_{\rm bar}}{a_0}\right),
\end{equation}
which is the mapping implemented in the v0.3.x reproducibility scripts. In this sense,
the RAR benchmark can be restated as a \emph{field-equation-compatible} effective law,
pending a microscopic derivation of $\mu$ (or $\nu$) from the Enchan defect model.
%===========================
% Chapter 7: Observation / Benchmarks
%===========================

\chap{Observational anchors and benchmarks}

\subsection*{Scope}

This chapter defines a minimal set of observational benchmarks that the effective
description must reproduce in the weak-field, galaxy-scale regime.
Only definitions are fixed here; no implementation details, code structure,
or procedural roadmap is included.

\subsection{Benchmark A: RAR / MDAR (multi-point relation)}
\label{sec:obs_rar}

For each radial point in each galaxy, define the observed centripetal acceleration
\begin{equation}
g_{\rm obs}(r) \equiv \frac{V_{\rm obs}(r)^2}{r},
\end{equation}
and the baryonic acceleration proxy from the standard mass-model decomposition
\begin{equation}
g_{\rm bar}(r) \equiv \frac{V_{\rm gas}(r)^2 + \Upsilon_d V_{\rm disk}(r)^2 + \Upsilon_b V_{\rm bul}(r)^2}{r},
\end{equation}
with consistent unit conversion.

The benchmark statement is the existence of a tight empirical mapping between
$g_{\rm bar}$ and $g_{\rm obs}$ across many radii and many galaxies, commonly
summarized by a one-parameter interpolation curve involving an acceleration scale
$a_0$.

\subsection{Benchmark B: BTFR (one galaxy = one point)}
\label{sec:obs_btfr}

Define a galaxy-level relation between baryonic mass $M_{\rm b}$ (stars + gas) and a
characteristic outer/flat rotation speed $V_{\rm f}$.
In log form,
\begin{equation}
x \equiv \log_{10}(V_{\rm f}/\mathrm{km\,s^{-1}}), \qquad
y \equiv \log_{10}(M_{\rm b}/M_\odot),
\end{equation}
and the benchmark is that a near-linear relation holds:
\begin{equation}
y = a + b\,x,
\end{equation}
with small intrinsic scatter.

A commonly used acceleration-scale diagnostic associated with BTFR is
\begin{equation}
a_{0,{\rm BTFR}} \equiv \frac{V_{\rm f}^4}{G\,M_{\rm b}},
\end{equation}
which provides a per-galaxy summary scale for comparison with the RAR-scale
normalization.

\subsection{Benchmark C: Rotation-curve prediction (shape test)}
\label{sec:obs_rcpred}

Using the baryonic mass-model components at each radius, compute $g_{\rm bar}(r)$.
Apply a fixed (globally shared) mapping to obtain a predicted response
$g_{\rm pred}(r)$, and define
\begin{equation}
V_{\rm pred}(r) = \sqrt{g_{\rm pred}(r)\,r}.
\end{equation}

The benchmark requirement is that a single fixed rule, without per-galaxy tuning,
produces nontrivial rotation-curve shapes across a large heterogeneous sample, and
that deviations exhibit interpretable systematics.

\subsection{Interpretation}

These benchmarks do not by themselves select a unique fundamental theory.
They serve as externally defined targets: an Enchan-derived forward model must
reproduce (A)--(C) within a unified mechanism and a consistent parameterization.


%===========================
% Bibliography
%===========================
\clearpage
\begin{thebibliography}{99}

\bibitem{McGaugh2016}
S.~S. McGaugh, F.~Lelli, and J.~M. Schombert,
``Radial Acceleration Relation in Rotationally Supported Galaxies,''
{\em Phys. Rev. Lett.} 117, 201101 (2016).

\bibitem{Lelli2019}
F. Lelli et al.,
``The Baryonic Tully--Fisher Relation for Different Galaxy Types,''
{\em Mon. Not. R. Astron. Soc.} 484, 3267 (2019).

\bibitem{Kibble1976}
T.~W.~B. Kibble,
``Topology of cosmic domains and strings,''
{\em J. Phys. A} 9, 1387 (1976).

\bibitem{BarriolaVilenkin1989}
M. Barriola and A. Vilenkin,
``Gravitational Field of a Global Monopole,''
{\em Phys. Rev. Lett.} 63, 341 (1989).

\end{thebibliography}

\end{document}
